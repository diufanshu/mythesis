% !Mode:: "TeX:UTF-8"
本文针对某汽车电子零部件企业装配车间调度问题进行研究,该车间具有多条同质装配线,采用多品种、批量、面向主机厂的装配方式,存在提高产线利用率、降低冗余度、均衡生产等提升空间。突破主机厂限制是有效的解决方案,这是一个多品种多装配线轮番调度优化问题。
建立了$2$个调度优化模型,模型1 适用于订单到达稳定的情况,模型2 适用于不稳定的情况。两个模型都以加权拖延时间和与加权完成时刻和为主体,采用决策系数将之结合,模型2 需要考虑插单,引入和定义了产线闲置、流水线利用率、均衡率等概念以适应其情况,根据多品种和多流水线的特点,设计了相应的约束条件。在模型求解算法中,定义了虚拟序列的概念,将之与调度规则及启发式策略结合设计了交替改进(Cycle Amend, Cyc) 和虚拟序列(Virtual List, Vtr) 类多个求解算法。计算实验结果表明:Vtr -- Tabu 算法适用于中小规模的不考虑插单情况的问题,且随着工期目标的重视而凸显改进效果,VVT 算法的求解结果在多种不同决策环境下都显示出了较高的质量和稳定性。所提算法求解该模型是有效果的,所建模型也适合该问题。
本文所提的多品种多流水线轮番调度模型及其算法确实进一步提高了流水线利用率、按时交付等性能,仍存在如加入混流生产的改进空间。

\keywordsc{多品种多装配线,调度模型,虚拟序列,算法设计}