% !Mode:: "TeX:UTF-8"
本文针对某汽车电子零部件企业装配车间调度问题进行研究,该车间具有多条同质装配线,采用多品种、批量、面向主机厂的装配方式,存在。。。


汽车电子零部件企业的装配车间普遍采用主机厂对应的专线来安排生产,存在诸多弊端,突破主机厂限制是有效的解决方案,这是一个多品种多装配线轮番调度优化问题。

建立了$2$个适用不同管理水平的调度模型,定义了虚拟序列的概念,将之与调度规则及启发式策略和设计了Cyc 和Vtr 类多个求解算法。计算实验结果表明:Vtr -- Tabu 算法适用于中小规模的不考虑插单情况的问题,且随着工期目标的重视而凸显改进效果,VVT 算法的求解结果在多种不同决策环境下都显示出了较高的质量和稳定性。所提算法求解该模型是有效果的,所建模型也适合该问题。虚拟序列的提出是本文的创新点,其应用范围很广,在建模方面可以进一步考虑混流生产。

\keywordsc{多品种多装配线,调度模型,虚拟序列,算法设计}