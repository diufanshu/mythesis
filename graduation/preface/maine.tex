% !Mode:: "TeX:UTF-8"
This thesis studied a scheduling problem in a assembly line of electronic parts for auto-mobile, however, the multi-type, large-batch, specialized assembly line by the main factories mode of production was taken, so that low utilization , high redundancy and unbalance production always emerged. A sound solution is to despecialize the assembly line, the implementation of this solution is a so called multi-type multi-assembly-line take-turn scheduling problem. In this study, 2 mathematical models is built for different levels of stability of the order arrival to apply. 
These 2 models were mainly objected by total weighted tardiness and total weighted completion, combined with decision factors. In model 2, idleness of line, utilization rate, balance rate were defined.
With the trait of the multi-type and multi-line, constrains were built.
With proposed concept of \textit{virtual list}, 
schedule rule and heuristic tactic,2 classes were designed, namely \textit{Cycle Amend}, \textit{Virtual List}. 
Experiment shows that Vtr -- Tabu algorithm suits for small and median scale problem, especially for none-job-inseting high weighted date-related problem, while high stability and quality schedule in various aspects can be obtain with VVT algorithm.

The proposed multi-type multi-assembly-line secheduling model and the solution algorithms do improve the utilization rate and delivery performance, while there is certainly room for improvement in the model.

\keywordse{~multi-type, multi-assembly-line, scheduling model, virtual list, algroithm design}