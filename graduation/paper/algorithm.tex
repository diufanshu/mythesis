% !Mode:: "TeX:UTF-8"
% !TEX root = ..\paper.tex
\newcounter{algor}%\newcounter{exam}
\theoremheaderfont{\heiti}
\newtheorem{algori}[algor]{算法}
\newcommand{\Step}[1]{\textbf{Step#1}}
\newcommand{\refa}[1]{\textbf{算法} \ref{#1}}
\chapter{算法设计}
%\section{复合分派规则}
%复合分派规则是综合了许多基本规则的一个表达式,例如ATC规则综合了WSPT规则和MS规则,每当有空闲处理单元时,所有待调度作业按\eqref{equ:orderindexexample}计算其排序指数,选出具有最大指数值的作业进行处理。
%\begin{equation}
%I_j(t) = \frac{wt_j}{p_j}\exp\left(-\frac{\max\{d_j - p_j - t, 0\}}{K\bar p}\right) \label{equ:orderindexexample}
%\end{equation}
%可以看出当$K \to \infty$时,$I_j(t) \to w_j/p_j$,此时ATC规则便转化为WSPT规则。当$K \to 0$时,若作业$j$将产生延期,即$\max(d_j - p_j -t , 0 ) = 0$,那么ATC规则也转化为WSPT规则,若作业不会产生延期,由于$d_j - p_j - t$的影响超过$w_j/p_j$,规则ATC将转化为MS规则。ATC规则可以较容易地应用到$Pm\mid\mid \sum w_jT_j$问题。
%其推广形式中的一种:ATCS 规则是WSPT 规则、MS 规则和SST 规则的复合规则,每当有空闲处理单元时,所有待调度作业按\eqref{equ:orderindexexample2}计算其排序指数,选出具有最大指数值的作业进行处理。
%\begin{equation}
%I_j(t) = \frac{w_j}{p_j}\exp\left(-\frac{\max\{d_j - p_j - t, 0\}}{K_1\bar p} - \frac{s_j}{K_2 \bar s}\right) \label{equ:orderindexexample2}
%\end{equation}
%其中:

%\begin{tabular}{ll}
%$K$ & 规则比例参数\\[5pt]
%$K_1$ & 与工期相关的规则比例参数\\[5pt]
%$K_2$ & 与切换准备时间的规则比例参数\\[5pt]
%$\bar p$ &剩余作业平均处理时间\\[5pt]
%$\bar s$ & 剩余作业平均切换准备时间
%\end{tabular}\\
\section{虚拟序列}
虚拟序列式本文提出一种新的邻域结构,用于例如禁忌搜索的邻域结构中,巧妙地避免了交替改进策略中流水线之间调整的问题,保证了解的改进调整全局性,使之更为合理,同时也不需要大量的运算次数。

所谓虚拟序列,即将所有流水线上的调度看作一个整体,所有订单都在这个序列上,其排列顺序由初始解的生成规则决定,也就是在调度安排时的记录序列$L$,并按先后顺序记该序列上的订单为$L_j, (j = 1,2,...,n)$。显而易见的是,任意流水线上两订单的先后顺序和它们在虚拟序列中的顺序是一样的。
举例来说,模型的初始解生成用到一定的规则,这个规则下,所有订单根据其排序指数进入空闲的流水线,那么从全局订单的角度来说,所有的订单也有一种先后安排顺序,每当一个订单安排进入某条流水线的时候,可以将其同时安排进入一条虚拟流水线,这条流水线中订单仅仅表示订单进入的先后,不用实际处理订单。进入不同流水线的订单都会同时进入同一个虚拟流水线。这条虚拟流水线上的调度即是虚拟序列。需要注意的是虚拟序列顺序和各流水线上的调度有很大的关系,但不是一一对应的,也就是说同样的虚拟序列可能会得到不同的实际调度安排,但每个实际调度只对应于1个虚拟序列。

虚拟序列上只有所有订单的先后信息,其订单的一种排序称为一种\textbf{虚拟调度}。虚拟调度的邻域设计和流水线内部调度类似,一次移动改变对于虚拟序列本身来说,就是该序列中两相邻的订单处理顺序交换,由于其非一一对应的特点,它体现在各流水线上的改变则要分为以下两种情况。
\begin{asparaenum}
\item 相邻订单$L_j, L_k$安排在同一条流水线$l$上进行处理
\suspend{asparaenum}

该情况下,现有调度的一种可行移动就是这两个相邻订单的交换。
\resume{asparaenum}
\item 相邻订单$L_j, L_k$分别安排在不同流水线$l, l'$上进行处理
\end{asparaenum}

该情况下,现有调度关于该两相邻订单的交换可以有$3$种移动:
\begin{inparaenum}
\renewcommand{\labelenumi}{\theenumi)}
\item 将这两个订单的位置互换,即$L_j$安排在流水线$l'$上订单$L_k$的原有位置,订单$L_k$亦然\label{item:situation1};
\item 将订单$L_j$重派到流水线$l'$上,其位置与$L_k$相邻,顺序和此次移到后的虚拟序列中两订单的顺序一致;
\item 将订单$L_k$重派到流水线$l$上,其位置与$L_j$相邻,顺序和此次移到后的虚拟序列中两订单的顺序一致。
\end{inparaenum}

后面两种情况在作移动时,不需要将订单交换对入栈禁忌列表,因为若再作一步该交换对的移动,不会回到之前的状态,而情况\ref{item:situation1}则只对两订单交换位置的移动禁忌。虚拟序列的禁忌搜索的基本策略是以虚拟序列中相邻订单对$(L_j, L_k)$交换作为一次移动,可以看出,一个调度的邻域容量不是一个定值,会随着调度结构的改变而改变。

需要注意的是禁忌列表内容,尤其是以下两种情况,皆由列表长度引起:
\begin{asparaenum}
\item 流水线之间相邻订单过度禁忌
\suspend{asparaenum}

这种情况是这样发生的,虚拟序列中的这两相邻订单作了一次流水线互换移动,其中的一个订单在该禁忌为出列表前,被重派到其他的流水线,而这两个订单的交换仍然处于禁忌状态。这种情况的出现会使得一些可行的移动无法进行,从而浪费了迭代次数和迭代时间。
\resume{asparaenum}
\item 流水线之内相邻订单过度禁忌
\end{asparaenum}
这种情况的发生和前面一种类似,虚拟序列中相邻的订单作了一次移动,属于流水线内的互换移动,然而其中的一个订单在该移动禁忌出列表前,重派到先前的流水线。这种情况的出现常常会导致算法出错,而且不易被发现。
为了防止这两种情况的发生,有其是第二种情况,每次迭代时要根据各流水线状况更新禁忌列表,消除这些过度禁忌。
\section{Vtr -- Tabu 算法}
运用虚拟序列的思想,结合禁忌搜索,可以得到求解多品种多装配线轮番装配调度优化模型求解算法如下:
\begin{algori}
Vtr -- Tabu 算法\label{alg:vtrtabu}
\begin{asparaenum}
\renewcommand{\labelenumi}{\bf Step\theenumi~}
\item 运用调度规则(如ATC、ATCS)建立流水线全局调度初始解,得到虚拟序列$L$及其初始调度$S^{(0)}$,并将其作为目前最优调度。设定禁忌搜索迭代次数$N_I$,设定列表长度$NL$,并置特赦调度$A = S^{(0)}$;
\item 置$S^{(1)} = S_{(0)}$,清空禁忌列表$TL$,置$k = 1$;
\item 在$L$所生成的邻域中,按顺序选取$(L_m, L_n)$,记当前调度为$S^-$,若$L_m, L_n$当前均安排在同一流水线的调度中,则执行\Step{4},否则执行\Step{5};
\item 交换订单对顺序,得到新的调度为$S^+$型;
\item 将订单$L_m$重派入流水线$l'$,得到调度为$S^{a+}$型,或将订单$L_n$重派入流水线$l$得到调度为$S^{b+}$型,或将订单$L_m, L_n$交换位置,得到调度为$S^+$型;
\item 更新虚拟序列中这两个订单的位置为$(L_m, L_n)$。
\item 检查禁忌列表中的订单对,若它们别安排在不同的流水线,则只对其交换位置的移动禁忌;若移动后的调度为特赦调度,一样认定为可行移动。计算$S^{(k)}$中所有可行移动组成的邻域,选取它们中具有最小函数值调度的移动,记该订单对为$(L_m^*, L_n^*)$,所得调度记为$S^*$,并置$S^{(k+1)} = S^*$;
\item 若相邻移动所得调度属于$S^+$型,则将$(L_m^*, L_n^*)$入栈禁忌列表,若列表容量已满,则按FIFO规则出栈最早的相邻移动,检查禁忌列表,删除过禁忌项;
\item 若$G(S^*) < G(S^{(0)})$,置$S^{(0)} = S^*$;
\item 置$k = k + 1$,若$k\le N_I$,执行\Step{3},否则终止算法,$S^{(0)}$为最终所得调度。
\end{asparaenum}
\end{algori}
\section{VVT 算法}
考虑插单的情况,Vtr -- Tabu 算法可以得到较有的结果,然而由于其邻域结构的特点,可能需要很大的迭代次数才能将解改进。采用变动邻域的策略可以人为切换邻域结构,放弃一些需要过多迭代次数的邻域结构,以减少计算时间,这样的综合策略称为VVT (变动邻域结构的虚拟序列禁忌搜索):
\begin{algori}
VVT 算法\label{alg:vvt}
\begin{asparaenum}
\renewcommand{\labelenumi}{\bf Step\theenumi~}
\item 运用调度规则(如ATC、ATCS)建立流水线全局调度初始解,得到虚拟序列$L$及其初始调度$S^{(0)}$,并将其作为目前最优调度,将其邻域集合$\overline{S^{(c)}}$中的调度按函数值的非减排列,记为$S_{[1]},S_{[2]},...,S_{[|S^{(c)}|]}$,置$i = 1$。设定禁忌搜索迭代次数$N_I$,设定列表长度$NL$,并置特赦调度$A = S^{(0)}$;
\item 若$i \le |S^{(c)}|$,置$S^{(1)} = S_{[i]}$,清空禁忌列表$TL$,置$k = 1$,否则终止算法;
\item 在$L$所生成的邻域中,按顺序选取$(L_m, L_n)$,记当前调度为$S^-$,若$L_m, L_n$当前均安排在同一流水线的调度中,则执行\Step{4},否则执行\Step{5};
\item 交换订单对顺序,得到新的调度为$S^+$型;
\item 将订单$L_m$重派入流水线$l'$,得到调度为$S^{a+}$型,或将订单$L_n$重派入流水线$l$得到调度为$S^{b+}$型,或将订单$L_m, L_n$交换位置,得到调度为$S^+$型;
\item 更新虚拟序列中这两个订单的位置为$(L_m, L_n)$。
\item 计算$S^{(k)}$中所有可行移动组成的邻域,选取它们中具有最小函数值调度的移动,记该订单对为$(L_m^*, L_n^*)$,所得调度记为$S^*$,并置$S^{(k+1)} = S^*$;
\item 若相邻移动所得调度属于$S^+$型,则将$(L_m^*, L_n^*)$入栈禁忌列表,若列表容量已满,则按FIFO规则出栈最早的相邻移动;
\item 若$G(S^*) < G(S^{(0)})$,置$S^{(0)} = S^*$;
\item 置$k = k + 1$,若连续$50$次采用没有更新$S^{(0)}$,则置$i = i+1$,执行\Step{2},否则若$k\le N_I$,执行\Step{3},否则终止算法,$S^{(0)}$为最终所得调度。
\end{asparaenum}
\end{algori}

VVT 算法改进了Vtr -- Tabu,使之不会陷入一个邻域花费过多运算次数,其中\Step{1}确定了待搜索的邻域结构,\Step{2} -- \Step{9}的操作和Vtr -- Tabu 算法基本一致,内嵌在邻域切换\Step{1}和\Step{10}中,该算法将得到调度结果$S$ 和相应的目标函数值$G(S)$。


