% !Mode:: "TeX:UTF-8"
% !TEX root = ..\paper.tex
\chapter{算法设计}
\section{复合分派规则}
复合分派规则是综合了许多基本规则的一个表达式,例如ATC规则综合了WSPT规则和MS规则,每当有空闲处理单元时,所有待调度作业按\eqref{equ:orderindexexample}计算其排序指数,选出具有最大指数值的作业进行处理。
\begin{equation}
I_j(t) = \frac{wt_j}{p_j}\exp\left(-\frac{\max\{d_j - p_j - t, 0\}}{K\bar p}\right) \label{equ:orderindexexample}
\end{equation}
可以看出当$K \to \infty$时,$I_j(t) \to w_j/p_j$,此时ATC规则便转化为WSPT规则。当$K \to 0$时,若作业$j$将产生延期,即$\max(d_j - p_j -t , 0 ) = 0$,那么ATC规则也转化为WSPT规则,若作业不会产生延期,由于$d_j - p_j - t$的影响超过$w_j/p_j$,规则ATC将转化为MS规则。ATC规则可以较容易地应用到$Pm\mid\mid \sum w_jT_j$问题。
其推广形式中的一种:ATCS 规则是WSPT 规则、MS 规则和SST 规则的复合规则,每当有空闲处理单元时,所有待调度作业按\eqref{equ:orderindexexample2}计算其排序指数,选出具有最大指数值的作业进行处理。
\begin{equation}
I_j(t) = \frac{w_j}{p_j}\exp\left(-\frac{\max\{d_j - p_j - t, 0\}}{K_1\bar p} - \frac{s_j}{K_2 \bar s}\right) \label{equ:orderindexexample2}
\end{equation}
其中:

\begin{tabular}{ll}
$K$ & 规则比例参数\\[5pt]
$K_1$ & 与工期相关的规则比例参数\\[5pt]
$K_2$ & 与切换准备时间的规则比例参数\\[5pt]
$\bar p$ &剩余作业平均处理时间\\[5pt]
$\bar s$ & 剩余作业平均切换准备时间
\end{tabular}\\
\section{虚拟序列}
定义虚拟序列
\section{Vtr -- Tabu 算法}

\section{VVT 算法}

