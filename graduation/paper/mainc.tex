% !Mode:: "TeX:UTF-8"
% !TEX root = ..\paper.tex
汽车电子零部件的装配采用多品种大批量生产方式,企业的装配车间普遍采用主机厂对应的专线来安排生产,存在如产线利用率低、冗余度高、生产不均衡等弊端。突破主机厂限制是有效的解决方案,引出了多品种多装配线轮番调度优化问题,建立了相应的调度模型,提出了虚拟序列的概念,将之与调度规则及启发式策略和设计了Cyc (交替迭代)类和Vtr (虚拟序列)类多个求解算法。使用Python 编写算法脚本进行计算实验,结果表明Vtr -- Tabu 算法适用于中小规模的不考虑插单情况的问题,且随着工期目标的重视而凸显改进效果,VVT 算法的求解结果在多种不同决策环境下都显示出了较高的质量和稳定性。

\keywordsc{多品种多装配线,调度模型,虚拟序列,算法设计}