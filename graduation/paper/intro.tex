% !Mode:: "TeX:UTF-8"
% !TEX root = ..\paper.tex
\chapter{引言}
多品种多流水线的装配调度是一种常见的车间调度问题,在不改变或较少改变现有生产设施的前提下,通过对装配生产线的合理组织与调度优化,实现多品种共线轮番装配,以最大限度地挖掘生产线的潜能。
有关调度问题的方法研究颇多,主要分为调度模型和调度算法两个方面。在建模方面,主要在于目标函数与约束的建立,
Xie Zhiqiang\cite{xie2010study}、唐勇智\cite{唐勇智2009基于聚类的}、陈伟\cite{陈伟2006生物信息学中的序列相似性比对算法}等研究了产品的分类,提出了许多确定产品簇的方法。
李斌\cite{李斌2009}、M.A. Adibi\cite{adibi2010multi}、刘文平\cite{刘文平2009}、杨本强\cite{杨本强2002}、李宏霞\cite{李宏霞2006}、B.J.V. da Silva\cite{da2014production}等设计了许多目标函数的建立方法以及多目标之间的权衡策略,A. Tharumarajah\cite{1998distributed}、李志娟\cite{李志娟2008}、P. Chutima\cite{chutima2014pareto}等提出的例如虚拟作业、阶段划分、学习因子、分布控制等方法,建立了适合其研究对象的模型约束。
算法方面,近来的研究热点以搜索算法为主,并呈现多种方法结合使用的趋势。
在解的搜索上,启发式方法在搜索局部最优时效果很好,而涉及到全局最优就需要考虑元启发算法。
陈琳\cite{陈琳2009}、Ham-Huah Hsu\cite{hsu1995fully}、Jia Shuai\cite{jia2014path}、G. Moslehi\cite{moslehi2011pareto}等改造禁忌搜索用于不同的调度问题,且得到了较好的求解结果。Rock Lin\cite{lin2012case}等通过设计合理的调度规则化简了计算进程,
朱有产\cite{朱有产2006}等通过改进经典的Dijkstra 搜索算法,引入空间向量,通过夹角参数,有效的快速跳出局部最优解
高丽\cite{高丽2012}等基于精英选择和个体迁移的多目标方法对遗传算法改进,得到了较好的多品种生产调度的Pareto 解集。
曹洪鑫\cite{曾洪鑫2006}、翁武熙\cite{翁武熙2012混合蚁群算法求解}等将多品种产品的混流装配顺序看作是商旅问题(TSP),其加工和换模时间即转化为路程,并通过遗传算法进行了100次迭代,较好的找到了较优解。

本文将借鉴上述建模和算法设计的方法,对某汽车电子有限公司装配车间的多品种多装配线轮番调度问题进行研究。


