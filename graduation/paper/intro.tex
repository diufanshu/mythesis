% !Mode:: "TeX:UTF-8"
% !TEX root = ..\paper.tex
\chapter{引言}
多品种多流水线的装配调度是一种常见的车间调度问题,在不改变或较少改变现有生产设施的前提下,通过对装配生产线的合理组织与调度优化,实现多品种共线轮番装配,以最大限度地挖掘生产线的潜能。
有关调度问题的算法研究颇多,而且随着目标函数的不同及目标数量的增多而逐渐呈现多元化,近来的研究热点以启发式算法为主,并呈现多种方法结合使用的趋势。
Xie Zhiqiang\cite{xie2010study} 等通过建立虚拟产品,将多品种问题转化为单品种问题,通过关键路径方法确定加工顺序后,根据各工位操作的特征运用不同的算法调度,然后进行整合,解决了单产品的柔性调度,最后添加作业间的约束,较好的解决了复杂的多品种调度问题。
唐勇智\cite{唐勇智2009基于聚类的}通过研究RBF-LBF 串联神经网络,改进K-means 聚类算法,提出了自适应的SA-K-means 算法,本课题的研究可以借鉴其思想,更为有效的将产品簇分类。
B.J.V. da Silva\cite{da2014production} 等通过航空行业的实例研究,考虑了人力的水平等级,学习影响及作业空间的限制,将飞机装配分成4个不同程度的阶段,各阶建立或增改约束模型,有效解决了含有邻接约束的装配调度问题。
A. Tharumarajah\cite{1998distributed}等考虑了基于行为的分布式控制,并将之用于装配调度模型中,有效地解决了一个3阶段4工作站的装配问题,与整数规划相比,基于行为的调度无论在运行时间或是适用规模上,都显著优于整数规划。
P. Chutima\cite{chutima2014pareto}等考虑了学习因子,提出了基于生物地域最优方法的方法,应用于两边装配线的混合模型,并引入了适应性机制,化解了最小化生产变化率、最小化总时间利用率、最小化调度序列换线时间三个矛盾,并使它们同时最优,提高了该元启发式方法的性能,有效的解决了多样性装配的调度。
在解的搜索上,启发式方法在搜索局部最优时效果很好,而涉及到全局最优就需要考虑元启发算法。
Jia Shuai\cite{jia2014path} 等通基于禁忌搜索算法,通过路线重建和回跳追踪方法进行排序决策,解决了多目标的柔性作业车间调度问题。
G. Moslehi\cite{moslehi2011pareto}等研究了柔性作业车间,提出了结合蚁群算法和邻域搜索算法的方法,解决了多目标的作业车间调度问题,得到了质量较高,计算时间合理的近似最优解,尤其在中、大型问题中更能体现优势,并且该方法的提升空间很大。
台湾学者Rock Lin\cite{lin2012case}等通过峰明机械厂的案例研究,引入作业划分概念和批量处理,提出了一个混合整数规划模型和3个启发式方法FBEDD、FBFS、RFBFS ,并通过计算实验证明FBEDD 在解决小型问题上较好,而对于中大型的问题,RFBFS 便具有更好的求解结果。
曹洪鑫\cite{曾洪鑫2006}等将多品种产品的混流装配顺序看作是商旅问题(TSP),其加工和换模时间即转化为路程,并通过遗传算法进行了100次迭代,较好的找到了较优解。本课题主要研究对象虽然不是混流装配,但混线时可以用到这个想法。

