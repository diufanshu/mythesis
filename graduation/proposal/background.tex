% !Mode:: "TeX:UTF-8"
% !TEX root = ..\proposal.tex
\chapter{课题研究背景与意义}
本文的研究对象是汽车零部件装配生产线,是典型的多品种小批量生产方式,并且在需求日益多样化的背景下,时常要根据产品调整生产。本文将从这种生产方式着手研究多品种产品的装配生产调度问题。

汽车装配大多采用同步装配流水线方式作业,将装配过程分为多个作业单元,并安排在流水线的相应工位上,
车体在移到中装配,各工位同时作业。以往,汽车装配工厂固定地生产一种或少数几种车型,采用大批量、规模化的生产。然而,随着技术的日新月异,客户需求的多样化,以及精益思想、环保节能观念的出现,汽车工业的生产模式不得不转变为面向订单的小批量、多品种的生产方式。因此,缩短交货期、提高资源利用率、降低生产成本、提高生产运作的灵活性,已成为保证企业市场竞争力的重要手段。

多品种装配是在基本不改变或较少改变现有生产设施的前提下,通过对装配生产线的合理组织与调度优化,实现多品种共线装配,以最大限度地挖掘生产线的潜能,向客户提供定制的个性化产品和服务。

生产调度就是组织执行生产进度计划的工作,作为一种决策形式,调度在制造业扮演者至关重要的角色,尤其在现今充满竞争的环境下,有效的生产调度已成为能在市场竞争下生存的必须。

从上个世纪50年代起,调度问题的研究就受到应用数学、运筹学、工程技术等领域科学家的重视,科学家们利用运筹学中的线性规划、整数规划、目标规划、动态规划及决策分析方法,研究并解决了一系列有代表意义的调度和优化问题\cite{徐俊刚2004}。

如今,随着计算机的发展,信息技术的成熟,许多过去只在理论层面上的调度算法,都可以通过计算机辅助得以验证与运用,给制造业的具体实力提升带来了可能。

提高竞争力的方法有很多,对于汽车工业,产品需求多样化和市场细分化,促使越来越多的制造商将多品种装配作为增强其竞争能力的有效手段。具体来说,对于汽车零部件,装配过程主要是以零部件的安装、紧固为主,其次是联接、压装和加注冷却液、制动液等液体以及整车质量检测的工序,有时还要根据用户意向选装。因此,合理安排装配产线,优化调度作业单元,对保证汽车装配质量,快速响应需求,提高汽车装配线的生产效率有着重要的现实意义。

而实现多产品装配不仅需要技术上的支持,也需要有理论来实践。虽然在Henry Gantt 的那个时代起,调度的理论研究就受到了制造业的关注,然而生产模式的转变,信息技术的出现,都使得一些过去经典的调度算法不再适用,这就需要我们来修正那些方法,或者发展新的算法,本课题便是以此为中心。

举例来说,随着调度问题的规模增大,人们发现即使通过计算机,有些问题的算法并不是有效的,因为它们的求解超出了可接受的运行时间。逻辑学家和计算机科学家通过研究这类问题,建立了复杂度理论,并称这些问题是$NP$问题,问题的复杂度会是随着问题规模增大呈现指数爆炸。如果能从汽车行业的零部件装配调度出发,研究出通用的算法,那就间接证明了$P=NP$,可谓意义非凡。然而本课题重点不在此,这个问题对于作者来说难度太大。

这样一来,有时得到最优调度或者最优解的成本就变得太高了,那么近似最优解便成了很好的选择。然而调度问题的算法本来就众多,求解近似最优解更是如此,不同的算法适用的情况也不尽相同。实践表明,寻找合适的调度方案对生产系统的运行有着显著的影响。因此,从多品种装配着手研究调度算法,对增加产品多样性,加快需求响应速度,加快提高企业的竞争力有着重要意义。

\chapter{课题研究的目标}
本课题的研究目标是,通过分析汽车零部件多品种装配车间的情况,找出和里的调度方案。同时建立通用的多品种装配车间的调度算法,进行计算检验,探讨其实用性。