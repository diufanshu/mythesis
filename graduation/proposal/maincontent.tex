% !Mode:: "TeX:UTF-8"
% !TEX root = ..\proposal.tex
\chapter{课题研究的主要内容}
\section{研究对象简介}
本课题的研究对象是某汽车电子有限公司装配车间,并由此展开的多品种装配调度问题。该公司装配车间有多条生产线,每条产线负责一种型号的汽车总装。
汽车零部件装配生产线,是典型的多品种小批量生产方式,并且在需求日益多样化的背景下,时常要根据产品调整生产。汽车装配大多采用同步装配流水线方式作业,将装配过程分为多个作业单元,并安排在流水线的相应工位上,
车体在移到中装配,各工位同时作业。以往,汽车装配工厂固定地生产一种或少数几种车型,采用大批量、规模化的生产。然而,随着技术的日新月异,客户需求的多样化,以及精益思想、环保节能观念的出现,汽车工业的生产模式不得不转变为面向订单的小批量、多品种的生产方式。因此,缩短交货期、提高资源利用率、降低生产成本、提高生产运作的灵活性,已成为保证企业市场竞争力的重要手段。

\section{研究对象存在的问题}
该厂汽车零部件装配调度的需要解决的问题有:
\begin{asparaenum}[(1)]
\item 装配线利用率低
\item 装配效率不高
\item 机器利用率低
\item 人员浪费较严重
\item 市场需求反应迟顿
\item ...
\end{asparaenum}

\section{主要研究内容}
根据实际调研和国内外研究现状,针对该厂汽车装配目前存在的一些问题和不足,提出以下研究内容:
\renewcommand{\labelenumi}{(\theenumi)}
\begin{asparaenum}
\item 识别瓶颈资源
\suspend{asparaenum}

在考虑多品种、小批量生产方式下瓶颈具有动态特性的基础上,将瓶颈资源定义为具有最长在制品队列长度或等待时间的机器,并由此建立生产物流瓶颈识别模型,解决瓶颈资源利用和控制的首要问题。
\resume{asparaenum}
\item 构建调度模型
\suspend{asparaenum}

通过分析平衡瓶颈资源效用,结合产品簇的分类和实际情况,确定有效的目标函数,在可接受的假设条件下设置相关约束构,建合理的调度模型。其中目标函数的确定即要考虑实际生产,又要兼顾求解的过程。
\resume{asparaenum}
\item 算法设计和实现
\end{asparaenum}

由于调度属于NP 问题,所以需要通过有效的搜索算法求解,算法的设计主要考虑可以在合理的时间内实现,并能得到最优或者质量较高的局部最优解
