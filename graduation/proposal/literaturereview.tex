% !Mode:: "TeX:UTF-8"
% !TEX root = ..\proposal.tex
% !TEX program = xelatex
\chapter{国内外文献综述}
这部分参考文献不要太少;要分类别来写,每类有数篇参考文献 ;国内外不一定要分开,但最后一定要分析现在的研究有什么优点、不足,对你的研究有什么帮助

对于多品种调度的研究,国内学者在调度的算法改造中有很多创新,在其研究的课题中体现出优良的性能,实用价值很大,对本课题的算法研究启发颇多。

李斌\cite{李斌2009}等提出了车间调度Multi-Agent 模型,以延期成本、设备利用率、综合调度性能等指标作为目标函数,并通过Lekin 软件和实例比较了不同调度规则下的效果。


高丽\cite{高丽2012}等基于精英选择和个体迁移的多目标方法对遗传算法改进,得到了较好的多品种生产调度的Pareto 解集。
陈琳\cite{陈琳2009}等将其撤装配车间简化成一个流水车间问题,并通过改进得到带有记忆的模拟退火算法,并引入禁忌搜索机制得到较好的近似最优解。
Jia Shuai\cite{jia2014path} 等通基于禁忌搜索算法,通过路线重建和回跳追踪方法进行排序决策,解决了多目标的柔性作业车间调度问题。
台湾学者Ham-Huah Hsu\cite{hsu1995fully}等研究了多机器人的装配单元,通过基于搜索算法的调度,并将之仿真。
G. Moslehi\cite{moslehi2011pareto}等研究了柔性作业车间,提出了结合蚁群算法和邻域搜索算法的方法,解决了多目标的作业车间调度问题,得到了质量较高,计算时间合理的近似最优解,尤其在中、大型问题中更能体现优势,并且该方法的提升空间很大。
M.A. Adibi\cite{adibi2010multi}等研究了随机作业和有机器抛锚可能的动态作业车间调度问题,通过经学习的神经网络,更新变邻域搜索算法的参数,在常见的分配规则下,较好的解决了该目标包括制造期和延迟的调度问题,并且其适用范围很广,其特点是神经网络的应用,很大程度上提高了搜索性能。
P. Chutima\cite{chutima2014pareto}等考虑了学习因子,提出了基于生物地域最优方法的方法,应用于两边装配线的混合模型,并引入了适应性机制,化解了最小化生产变化率、最小化总时间利用率、最小化调度序列换线时间三个矛盾,并使它们同时最优,提高了该元启发式方法的性能,有效的解决了多样性装配的调度。
刘文平\cite{刘文平2009}将汽车装配的多种订单产品序列优化看作约束满足的调度模型,通过邻域搜索算法中的Memetic 算法,优化了混合品种装配线调度。
曹洪鑫\cite{曾洪鑫2006}等将多品种产品的混流装配顺序看作是商旅问题(TSP),其加工和换模时间即转化为路程,并通过遗传算法进行了100次迭代,较好的找到了较优解。本课题主要研究对象虽然不是混流装配,但混线时可以用到这个想法。讲到(TSP),翁武熙\cite{翁武熙2012混合蚁群算法求解}采用了结合蛙跳算法的新型智能算法,较好的改进了蚁群算法,值得借鉴。



李宏霞\cite{李宏霞2006}等载荷考虑了物料配送能力,运用FCFS 规则的相关算法\cite{lo2002job}提出了一种操作性较强的调度模型,较好的解决了多品种变批量的装配调度问题。


Xie Zhiqiang\cite{xie2010study} 等通过建立虚拟产品,将多品种问题转化为单品种问题,通过关键路径方法确定加工顺序后,根据各工位操作的特征运用不同的算法调度,然后进行整合,解决了单产品的柔性调度,最后添加作业间的约束,较好的解决了复杂的多品种调度问题,对于简单的多品种调度问题,甚至可以不用加入作业间的约束。




台湾学者Rock Lin\cite{lin2012case}等通过峰明机械厂的案例研究,引入作业划分概念和批量处理,提出了一个混合整数规划模型和3个启发式方法FBEDD、FBFS、RFBFS ,并通过计算实验证明FBEDD 在解决小型问题上较好,而对于中大型的问题,RFBFS 便具有更好的求解结果。


国外学者在研究调度算法时,一般会将多种方法结合使用,而且新方法较多,发展空间大,对本课题的研究很有借鉴意义。


B.J.V. da Silva\cite{da2014production} 等通过航空行业的实例研究,考虑了人力的水平等级,学习影响及作业空间的限制,将飞机装配分成4个不同程度的阶段,各阶建立或增改约束模型,有效解决了含有邻接约束的装配调度问题。
A. Tharumarajah\cite{1998distributed}等考虑了基于行为的分布式控制,并将之用于装配调度模型中,有效地解决了一个3阶段4工作站的装配问题,与整数规划相比,基于行为的调度无论在运行时间或是适用规模上,都显著优于整数规划。

杨本强\cite{杨本强2002}运用线性规划理论,建立了汽车流水线均衡生产模型,并通过一个启发式搜索算法来探寻解,思路简单,容易实现,本课题的微观调整可以借鉴其思想。


李志娟\cite{李志娟2008}等通过研究高校排课问题,在有效、交错、分散、固定、优先的原则下,设计了一个基于规则的算法,并在产生冲突时进行回溯,得到较好的课程表,其设计思想可以用到本课题中,有其是在多种目标下,设计相应的规则算法。

朱有产\cite{朱有产2006}等通过改进经典的Dijkstra 搜索算法,引入空间向量,通过夹角参数,有效的快速跳出局部最优解,得到最短路径问题的全局最优,对本课题的解的搜索有启发意义,可以将之结合入粒子群算法,重新定义其方向参数以改进。


陈伟\cite{陈伟2006生物信息学中的序列相似性比对算法}通过Smith-Waterman, FASTA 和BLAST 三个局部对比算法,较好的找到了相似性较高的DNA 序列,对于本课题中的产品簇归类有很高的借鉴意义。

唐勇智\cite{唐勇智2009基于聚类的}通过研究RBF-LBF 串联神经网络,改进K-means 聚类算法,提出了自适应的SA-K-means 算法,本课题的研究可以借鉴其思想,更为有效的将产品簇分类。