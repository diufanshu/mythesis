% !Mode:: "TeX:UTF-8"
% !TEX root = ..\Literature_Translation.tex
\kchapter{计算结果}
使用建模语言GAMS(v23.0)建立该数学模型,并用CPLEX 11.2.1.0 使用4核求解。计算实验在一台处理器为Intel i7(2.8GHz, 12GB)的PC 机上运行,所有示例实验时间设为10 h。

这些示例包括15个活动,通常这些活动在工作台上的操作时间要长于夹具。
为了保护公司数据的隐私性,持续时间的单位(t.u.)不表示实际时间,因为其乘上了一个常数。需要注意的是,每架飞机需要两整套装配子集,因此,活动数目被扩大成30个活动。

\begin{table}[htbp]
  \centering
  \caption{Add caption}
    \begin{tabular}{ccccccc}
    \toprule

子集 &子集的零件 &活动(j) &\multicolumn{1}{m{20mm}}{夹具中的工作站} &\multicolumn{1}{m{20mm}}{夹具中的持续时间(mj)} &  \multicolumn{1}{m{20mm}}{工作台上的持续时间(qj)} &  \multicolumn{1}{m{20mm}}{前继活动(A(j,k)) }\\
\midrule
    1     & 1     & 1     & 1     & 5     & 7     & - \\
    1     & 1     & 2     & 1     & 4     & 47    & 1 \\
    2     & 1     & 3     & 1     & 5     & 7     & - \\
    2     & 1     & 4     & 1     & 4     & 47    & 3 \\
    1     & 2     & 5     & 2     & 15    & 29    & - \\
    1     & 2     & 6     & 2     & 9     & 71    & 5 \\
    2     & 2     & 7     & 2     & 15    & 29    & - \\
    2     & 2     & 8     & 2     & 9     & 71    & 7 \\
    1     & 3     & 9     & 3     & 11    & 40    & - \\
    2     & 3     & 10    & 3     & 11    & 40    & - \\
    1     & 4     & 11    & 4     & 12    & 40    & - \\
    1     & 4     & 12    & 4     & 6     & 40    & 11 \\
    2     & 4     & 13    & 4     & 12    & 40    & - \\
    2     & 4     & 14    & 4     & 6     & 40    & 13 \\
    1     & 5     & 15    & 5     & 8     & 30    & - \\
    1     & 5     & 16    & 5     & 6     & 40    & 15 \\
    2     & 5     & 17    & 5     & 8     & 30    & - \\
    2     & 5     & 18    & 5     & 6     & 40    & 17 \\
    1     & 6     & 19    & 6     & 12    & 17    & - \\
    1     & 6     & 20    & 6     & 6     & 40    & 19 \\
    2     & 6     & 21    & 6     & 12    & 17    & - \\
    2     & 6     & 22    & 6     & 6     & 40    & 21 \\
    1     & 7     & 23    & 7     & 7     & 12    & - \\
    1     & 7     & 24    & 7     & 7     & 30    & 23 \\
    2     & 7     & 25    & 7     & 7     & 12    & - \\
    2     & 7     & 26    & 7     & 7     & 30    & 25 \\
    1     & 8     & 27    & 8     & 10    & 61    & - \\
    1     & 8     & 28    & 8     & 12    & 66    & 27 \\
    2     & 8     & 29    & 8     & 10    & 61    & - \\
    2     & 8     & 30    & 8     & 12    & 66    & 29 \\
    \bottomrule
    \end{tabular}%
  \label{tab:addlabel}%
\end{table}%
