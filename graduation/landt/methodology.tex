% !Mode:: "TeX-UTF-8"
% !TEX root = ..\Literature_Translation.tex
\chapter{作业调度相关理论}
无论从技术或者是应用的角度来说,调度方案的制定都具有一定难度。技术上来说,包括目标的组合以及模型的随机性,从应用上来说,包括模型的适用准确程度以及输入输出数据的可靠\cite{pinedo}。

\section{框架及符号说明}
\subsection{确定型模型}
确定型调度模型假设工作数量和机器数量有限,作业数量记为$n$并以下标$j$指代,机器数量记为$m$并以下标$i$指代,用$(i,j)$表示作业$j$在机器$i$上的操作或者处理。常用相关数据如下:
\renewcommand{\descriptionlabel}[1]{\heiti{#1}}
\begin{compactdesc}
\item[加工时间$(p_{ij})$]作业$j$在机器$i$上的加工时间,如果作业的处理时间不依赖于机器或者是单机处理,那么可以记为$p_j$。
\item[处理速度$(v_{ij})$]机器$i$的处理作业$i$的速度。
\item[提交日时$(r_j)$]也称作准备日期,是系统可以开始运行的最早时间。
\item[工期$(d_j)$]作业的承诺发运或完成时间。作业虽然可以在工期后完成,不过可能产生相应惩罚。如果工期必须满足,则称为最后期限$\bar{d_j}$。
\item[权重$(w_j)$]作业的优先系数,说明其在系统的重要程度,例如持有成本或附加价值。
\end{compactdesc}

一个调度问题可以用$\alpha\ |\ \beta\ |\ \gamma$描述,其中$\alpha$描述机器环境,$\beta$描述加工特征和约束细节,$\gamma$描述最小化的目标。

具体的$\alpha$环境有:
\begin{compactdesc}
\item[单机$(1)$]是一个最简单的特殊机器环境。
\item[并行同速机$(Pm)$]$m$台并行的处理速度相同的机器。
\item[并行异速机$(Qm)$]$m$台并行的但处理时间不同的机器。
\item[并行无关机$(Rm)$]是前面一种机器环境的推广,如果机器的处理速度独立于作业,则为前面的情况。
\item[流水车间$(Fm)$]
\end{compactdesc}
\section{小结}
本课题的主要技术关键点的比较分析和实现方法