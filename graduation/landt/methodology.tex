% !Mode:: "TeX-UTF-8"
% !TEX root = ..\Literature_Translation.tex
\chapter{作业调度相关理论}
无论从技术或者是应用的角度来说,调度方案的制定都具有一定难度。技术上来说,包括目标的组合以及模型的随机性,从应用上来说,包括模型的适用准确程度以及输入输出数据的可靠\cite{pinedo}。

\section{框架及符号说明}
\subsection{确定型模型}
确定型调度模型假设作业数量和机器数量有限,作业数量记为$n$并以下标$j$指代,机器数量记为$m$并以下标$i$指代,用$(i,j)$表示作业$j$在机器$i$上的操作或者处理。常用相关数据如下:
\renewcommand{\descriptionlabel}[1]{\heiti{#1}}
\begin{compactdesc}
\item[加工时间$(p_{ij})$]作业$j$在机器$i$上的加工时间,如果作业的处理时间不依赖于机器或者是单机处理,那么可以记为$p_j$。
\item[处理速度$(v_{ij})$]机器$i$的处理作业$i$的速度。
\item[提交日时$(r_j)$]也称作准备日期,是系统可以开始运行的最早时间。
\item[工期$(d_j)$]作业的承诺发运或完成时间。作业虽然可以在工期后完成,不过可能产生相应惩罚。如果工期必须满足,则称为最后期限$\bar{d_j}$。
\item[权重$(w_j)$]作业的优先系数,说明其在系统的重要程度,例如持有成本或附加价值。
\end{compactdesc}

一个调度问题可以用$\alpha\ |\ \beta\ |\ \gamma$描述,其中$\alpha$描述机器环境,$\beta$描述加工特征和约束细节,$\gamma$描述最小化的目标。

具体的$\alpha$域有:
\begin{compactdesc}
\item[单机$(1)$]是一个最简单的特殊机器环境。
\item[并行同速机$(Pm)$]$m$台并行的处理速度相同的机器。
\item[并行异速机$(Qm)$]$m$台并行的但处理时间不同的机器。
\item[并行无关机$(Rm)$]是前面一种机器环境的推广,如果机器的处理速度独立于作业,则为前面的情况。
\item[流水车间$(Fm)$]每项作业必须按相同的顺序,经过串行的$m$台机器的处理。作业在其中一台机器上完工后,需加入下一台机器的队列中,通常这些队列遵守先入先出(FIFO)规则。
\item[柔性流水车间$(FFc)$]是流水车间和并行机环境的一般化,由$c$个串行阶段组成,每个阶段由多台并行同速机组成。每项作业需要按相同顺序经过各阶段的处理,可以选取同阶段的任何一台机器处理,其队列遵守先来先服务(FCFS)规则。
\item[加工车间$(Jm)$]在有$m$台机器的加工车间,每项作业都有其预定的加工路径,有些加工车间的作业只经过其中的机器至多一次,有些加工车间则不然。
\item[柔性加工车间$(FJc)$]柔性加工车间是加工车间和并行机环境的一般化,如同柔性流水车间模式,用$c$个含有多台并行机阶段的作业中心代替$m$台机器即可。
\item[开放车间$(Om)$]作业加工路径没有限制,可由调度者决定各作业的加工路线。
\end{compactdesc}

具体的$\beta$域有:
\begin{compactdesc}
\item[提交日期$(r_j)$]在$\beta$域中,如果出现提交日期,则作业不可以在其之前开始。
\item[确定顺序准备时间$(s_{jk})$]是按顺序排的在作业$j$、$k$的准备时间,如果作业$k$是第一项作业,那么$s_{0k}$表示作业$k$的准备时间,如果作业$j$是最后一项作业,那么$s_{j0}$表示作业$j$的清理时间。
\item[中断$(prmp)$]中断意味着作业可以在完成前从机器去下,机器可以用于处理其他的作业,被中断的作业之后只要完成剩下的作业即可。
\item[优先约束$(prec)$]在单机或并行机环境下,每一项作业开始前,另一项作业必须完成。几种优先约束的特殊形式:每项作业最多有一个后继和前驱,称为链式,最多有一个后继,称为入树,最多有一个前驱,称为出树。
\item[故障$(brkdwn)$]
\item[机器适用限制$(M_j)$]
\item[排列$(prmu)$]
\item[阻塞$(block)$]
\item[无等待$(nwt)$]
\item[再循环$(recrc)$]
\item[作业簇$(fmls)$]同簇作业可能有不通的处理时间,但在同机器处理的作业间不需要准备时间,但作业簇切换时,需要考虑准备时间,如果准备时间与作业簇$h$和$g$有关,则记为$s_{gh}$,若只和将要处理的作业簇$h$相关,则记为$s_h$,若和作业簇无关,则记为$s$。
\item[批量处理$(batch(b))$]有些机器可以同时处理$b$项作业,由于各作业的处理时间可以不同,批量处理的时间由其中处理时间最长的作业决定。
\end{compactdesc}

\subsection{随机型模型}
\section{小结}
本课题的主要技术关键点的比较分析和实现方法