% !Mode:: "TeX:UTF-8"
% !TEX root = ..\Literature_Translation.tex
\stepcounter{app}
\setcounter{figure}{0}
\setcounter{table}{0}
\newpage
\begin{Abstract}
\chapter*{航空工厂的装配夹具生产调度}\addcontentsline{toc}{section}{航空工厂的装配夹具生产调度}
\begin{center}
\vspace{2mm}
{
 {\xiaosi Bruno Jensen Virginio da Silva$^a$,  Reinaldo Morabito$^a$, Denise Sato Yamashita$^a$,\\ Horacio Hideki Yanasse$^b$}

 {\xiaowu $^a$Departamento de Engenharia de Produção, Universidade Federal de São Carlos, Brazil\\
 $^b$Nati Instituto de Ciência e Tecnologia, Universidade Federal de São Paulo, Brazil}
}
\end{center}
{\songti
\noindent \xiaowu\textbf{摘要:}我们将在本文研究在航空工厂的装配生产调度问题。飞机零件需要在夹具上生产制造,这在飞机制造上是很常见的,而且会由多个
工作站组成。由于夹具的物理工件小,当一个工作站在工作的时候,与其相邻的工作站便无法工作,这些约束称为邻接约束。该装配夹具调度问题的研究背景是,将员工的学习进程分为4个主要限制阶段(或时期)。最优化求解器和建模语言将运用到各阶段的数学建模和应用。该方法的计算实验将被运行在一个巴西航空工厂的案例研究,并可以得到该方法比现行方法要好的结论。

\keywords{生产调度、装配夹具、航空工厂、邻接约束}
}
\end{Abstract}
% !Mode:: "TeX:UTF-8"
% !TEX root = ..\Literature_Translation.tex
\kchapter{引言}
我们将在本文描述一个产生于飞机制造的装配夹具的生产调度问题。
我们对受邻接约束下,为了装配而用到夹具的这部分特殊飞机零件的生产特别感兴趣。夹具是用来夹紧工件或作为支撑的设备,专为适用特定零件或形状而建。
夹具的主要作用是定位,在某些情况下也用于整个加工操作中或其他工业过程的工件保持(Niu,1988;Drake,1989;Howe,2004)。
由于空间限制,夹具相邻的工作站不能同时开工,产生了邻接约束。
装配一个飞机零件至少需要2个步骤,其中之一需要在夹具上完成,另一个互补的装配操作将在工作台上进行。根据装配所需的零件,这两个操作需要重复操作。


