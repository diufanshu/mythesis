% !Mode:: "TeX:UTF-8"
%\setcounter{chapter}{0}
\stepcounter{app}
\begin{Abstract}
\chapter*{某装配车间批调度案例研究}\addcontentsline{toc}{section}{某装配车间批调度案例研究}
\begin{center}
\vspace{2mm}
{
 {\xiaosi Rock Lin, Ching-Jong Liao}

 {\xiaowu Department of Industrial Management, National Taiwan University of Science and Technology, Taipei 106, Taiwan}
}
\end{center}
{\songti 
\noindent \xiaowu\textbf{摘要:}本文探讨了在某机械工厂装配车间内的生产调度问题。该装配工艺包含两个工段:在工段1,所需零件在一批同速机上同时装配,并且这些同速机的准备时间也相同;装配完成的部件进入工段2进,并在不同的异速机上进行系统集成组装。同速机和异速机在切换生产产品簇的时候都需要考虑换线时间。本文建立了一个混合整数规划(MIP)模型以求解小型问题,并提出了用于求解中、大型问题的三个启发式方法。经过计算检验,相比较其余两个方法,其中一个利用滚动时域调度策略的整批产品簇排序启发式组合的方法(RFBFS),在解决问题方面有较高质量。实践表明,RFBFS方法确实显著优于现行方法。

\keywords{混合整数规划、作业划分、批量作业、产品簇调度}
}
\end{Abstract}
\kchapter{引言}
在机械制造工厂中,经常会遇到批调度问题,我们将在本文具体研究工厂中的两阶段组装车间的调度。这个问题与Yokoyama 等的论文中所描述的两阶段组装车间调度问题类似:工段1,生料组装成产品部件;工段1,这些部件装配成产品。然而,由于技术与工艺十分复杂,生料的组装不仅困难,而且成本高,这便是通常情况下,企业将生料组装外包(或直接采购组装件)的原因。因此,我们为这个装配车间构建成一个两阶段流水车间模型,在工段1进行模块化装配,在工段1进行系统集成。

随着20年的技术不断革新,需求的多样性也水涨船高,产品便不断向着多功能、高性能的要求发展,这使得机械化生产变得日益复杂。如此背景下,工厂的车间管理不得不处理好下列调度问题:
\begin{compactenum}[(1)]
\item 如何缩短生产周期?
\item 如何减少在制品(WIP)?
\item 如何实现及时交货?
\item 如何处理紧急插单?
\end{compactenum}