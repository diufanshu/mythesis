% !Mode:: "TeX:UTF-8"
% !TEX root = ..\Literature_Translation.tex
\chapter{总结}

调度问题本身具有高复杂性、多约束性、多目标性,在多品种小批量生产方式下,由于产品品种多、批量小、零件多等特点,增加了调度的复杂程度。多品种的装配调度问题是一种常见的车间调度问题,对于汽车零件,又有其独有的工艺特点。
本文针对多品种装配车间调度问题分析了部分国内外相关的研究。

国内外学者在调度的算法改造中有很多创新,在其研究的课题中体现出优良的性能,实用价值很大,对本课题的算法研究启发颇多,在研究调度算法时,一般会将多种方法结合使用,而且新方法较多,发展空间大,对本课题的研究很有借鉴意义。主要研究集中在将产品分类或者确定产品簇,有关建模设计的研究,和解的搜索上,这在多品种的装配问题来说很重要,启发式方法在搜索局部最优时效果很好,而涉及到全局最优就需要考虑元启发算法。这对本课题的研究有启发意义,更加明确课题的研究方向。将产品分成产品簇是调度算法设计的前提,然后需要设计并改善目标函数,使之适合研究对象的实际,再改造搜索过程,优化实现算法。

随后,本文简要说明了调度的相关理论,确立了确定型和随机型的模型框架和概念。有关调度问题的算法众多,而各方法的适用对象亦不尽相同。对于多品种的装配调度,需要考虑的目标较多,许多目标间可能存在矛盾关系,需要有合理的解决办法。另一方面,合理的约束条件可以使问题更符合实际,但同时也增加了难度。本文之后选取了3个与本课题相关度较高的方法,FBFS,VNS,PSO,涉及目标函数的优化及解的搜索。FBFS 算法的特点是将作业按簇分成批次,可以有效减少作业簇准备时间,适合多品种的生产调度,然而该方法只按优先级排序作业,按批次的特点安排生产,可能会导致部分作业滞后完成而部分作业却提前完成。这个方面可以通过例如滚动时域的方法将其改善。VNS 可通过扰动来使之跳出局部最优,虽然这个方法在解决多目标优化问题的时候效果较好,然而随着目标的复杂程度增加,搜索空间大小将呈现指数增长,求解困难增加。可以通过一些方法,如分支定界或有限差异搜寻等提高搜索质量。同时,也可通过神经网络的方法来得到各目标间的权衡系数,以提高目标函数质量。POS 技术只需初级和简单数学操作即可完成,通过粒子的速度和位置两个部分,形象了搜索过程,能较快达到近似最优解,所需时间和成本都较低,能够较好处理目标很多的柔性流水车间调度,然而其能处理的问题规模较小。对于能拆分成小型问题的中大型问题,此方法就很有优势,此外要关注起收敛速度。本文同时列举了两个启发式搜索算法的案例,案例1将多品种装配调度问题可以看作是典型TSP,虽然具有NP 计算复杂性,但可以通过合理的遗传算法进行求解,找到质量较高的排序方案。在遗传算法的设计中,交叉和变异是跳出局部解的关键,参数的设置需要考虑计算时间和解的质量的权衡。案例2表明变动邻域搜索算法可以应对多目标问题,尤其对动态调度显示出优势,然而其使用的限制还是比较苛刻的,不过其变动策略为求解复杂调度问题提供了新的搜索思路,而且从结果上来看,变动邻域搜索算法的性能很好。