% !Mode:: "TeX:UTF-8"
% !TEX root = ..\thesis.tex
\chapter{装配生产线分析及模型建立}
本章将对装配线生产进行分析,指出现有计划安排和调度存在的主要问题,然后针对这些问题,提出合理解决方案,并对其进行分析和数学建模。
\section{生产线分析}
课题研究对象是该汽车电子公司的装配生产线,是典型的流水车间,每条流水线负责单一品种的总装流程。不同汽车的总装工艺类似,品种的区别在于其配置与型号等,体现在生产线上为零部件、机器设备、装配流程的区别,
不同订单间还存在批量的差别。

生产线分析包括上述这些差别及其产生的影响或效果,具体描述从订单、装配到交付的流程,并分析现行调度方案的一些指标。

\subsection{流程描述}

\subsection{现行调度方案}
采用

\subsection{主要问题}
指出现行调度的主要问题,并分别对其进行改进。

根据以上现行调度的情况分析,可以归纳其主要存在问题如下:
\renewcommand{\labelenumi}{(\theenumi)}
\begin{asparaenum}
\item 产线利用率低
\suspend{asparaenum}

reason1...
\resume{asparaenum}
\item 装配效率不高
\suspend{asparaenum}

reason2...
\resume{asparaenum}
\item 机器利用率低
\suspend{asparaenum}

reason3...
\resume{asparaenum}
\item 人员浪费严重
\suspend{asparaenum}

reason2...
\resume{asparaenum}
\item 市场反应迟钝
\end{asparaenum}

\section{模型建立}
运用混流生产中的品种均衡方法可以有效解决生产平衡问题,有关混流生产的研究也十分众多,理论相对较为成熟。
然而实施均衡化生产需要有较高的管理水平,不太适用于本课题的研究对象。因此,本节将结合实际情况,权衡投入和效益,建立适合本课题研究对象的混线装配生产模型。
\subsection{方案改进}


\subsection{符号说明}

\subsection{相关假设}

\subsection{目标函数}

\subsection{约束条件}


\section{小结}