% !Mode:: "TeX:UTF-8"
% !TEX root = ..\thesis.tex
\chapter{装配生产线分析及模型建立}
本章将对装配线生产进行分析,指出现有计划安排和调度存在的主要问题,然后针对这些问题,提出合理解决方案,并对其进行分析和数学建模。
\section{生产线分析}
课题研究对象是该汽车电子公司的装配生产线,是典型的流水车间,每条流水线负责单一品种的总装流程,

\subsection{现行调度方案}
采用

%%%%%%%%%%%%%%
该公司装配车间有多条作业流水线,每条流水线负责某一种品牌单项产品的总装流程,生产较为固定,产线间的工装设备及流程相似但不相同,不同客户的订单基本不在相同的产线上装配。

装配作业采用同步装配流水线方式,将装配过程分为多个作业单元,并安排在流水线的相应工位上,车体在移动中装配,各工位同时作业。不同产线间会出现相同的作业内容,但考虑到装配过程的连续性,需要重复安排人员及设备。

该公司采用面向订单生产,客户源较为稳定,其需求特点使得装配作业呈现为多品种小批量的生产方式。在没有订单或者订单较少时,为了不让生产线停下来,需要进行工厂内部的计划生产,而订单较多时需要加班作业。

\section{产线调度现状}
当前该公司装配车间采用专线生产的方式,即客户的订单在其专用的流水线上进行生产作业,当同一客户有多个订单下达时,按照先到先服务(FCFS)的规则进行调度安排,多条产线并行作业互不干扰。
%%%%%%%%%%%%%%%%%%%%%%%%%%%%%%%%%%%

\subsection{主要问题}
根据以上现行调度的情况分析,可以归纳其主要存在问题如下:
\renewcommand{\labelenumi}{(\theenumi)}
\begin{asparaenum}
\item 产线利用率低
\suspend{asparaenum}

reason1...
\resume{asparaenum}
\item 装配效率不高
\suspend{asparaenum}

reason2...
\resume{asparaenum}
\item 机器利用率低
\suspend{asparaenum}

reason3...
\resume{asparaenum}
\item 人员浪费严重
\suspend{asparaenum}

reason2...
\resume{asparaenum}
\item 市场反应迟钝
\end{asparaenum}

哈哈
\subsection{方案改进}

\section{模型建立}
\subsection{符号说明}

\subsection{相关假设}

\subsection{目标函数}

\subsection{约束条件}


\section{小结}