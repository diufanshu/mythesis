% !Mode:: "TeX:UTF-8"
% !TEX root = ..\thesis.tex
\chapter{多品种装配车间调度建模}
上一章分析了现行调度存在的问题并提出了改进方案设计,而该方案的实现是一个多品种多装配线的调度问题,本章将对该问题进行具体分析与建模,然后根据该问题的特点,建立相应的调度优化模型。


\section{多品种多装配线轮番装配调度优化模型}
根据改进调度方案的特点,可以建立相应的调度优化模型,以下简称模型1。


根据上一章的改进设计,对所需生产的订单进行排列组合,较为均匀地安排在各流水线上,进行混合流水线轮番装配,以期获得均衡的流水线利用率、减少换线时间浪费、缩短完工时间、降低生产成本等。需要注意的是这种生产方式和混流生产的区别,在混流生产是将不同的订单中的产品根据一定比例混合,在流水线上进行生产,使生产均衡化,其研究单位为订单中的作业。而多流水线混线装配的研究单位是各订单,其生产方式打破现行的订单在固定流水线上生产的限制的策略,从而合理利用各流水线产能资源。

产品订单确定其所需产品的数量(订单批量),订单中的每个产品可以看作为作业,流水线上的各装配工位或机器看作处理单元,由于本课题的研究主要内容较少涉及具体的装配工艺,故产品订单也确定了作业处理所需的处理单元的数量、种类以及顺序,流水线上的订单处理可以看作是各作业按固定顺序经过线上的处理单元进行处理。

\subsection{基本符号说明}
为了方便问题描述,需要说明基本符号如下,其中涉及的时间变量研究对象为系统时间:\\[3pt]
\begin{supertabular}{ll}
$n$ & 订单数量\\
$m$ & 流水线数量\\
$j$ & 订单标记,$j = 1,2,...,n$\\
$N$ & 所有订单集合$\{ j\mid j \in \mathbb{Z}, 1\le j \le n  \}$\\
$l$ & 流水线标记,$l = 1,2,...,m$\\
$S_l$ & 流水线$l$上的订单调度\\
$|S|$ & 调度$S$的订单数量\\
$\overline S$ & 调度$S$中的订单集合\\
$l_k$ & 调度$S_l$的第$k$项订单标记,,$k = 1,2,...,|S_l|$\\
$d_j$ & 订单$j$的交货时刻(工期)\\
$t$ & 生产系统时间\\[3pt]
\end{supertabular}

一个调度问题可以由三元组$\alpha \mid \beta \mid \gamma$表示,$\alpha$域描述单一处理单元环境,$\beta$域包含加工特征和约束的细节,$\gamma$域描述其目标\cite{pinedo}。

\begin{asparaenum}
\item 基本$\alpha$域
\suspend{asparaenum}

\begin{supertabular}{ll}
$Pm$ & 同速并行机\\
$Fm$ & 流水车间\\
\end{supertabular}
\resume{asparaenum}
\item 基本$\beta$域
\suspend{asparaenum}

\begin{supertabular}{ll}
$r_j$ & 订单$j$到达流水线系统时刻,是其最早可开始时刻\\
$s_j$ & 订单$j$开始之前所需的切换(开工)准备时间\\
\end{supertabular}
\resume{asparaenum}
\item 基本$\gamma$域
\end{asparaenum}

$\gamma$域涉及目标函数,一般调度问题需要考虑最小化目标函数,常见的目标函数为订单$j$的完成时刻$C_j$,为订单$j$离开系统的时刻。订单$j$的迟滞可以定义为:
\[
L_j = C_j - d_j
\]
进一步可定义其延迟和提前:
\begin{align*}
T_j & = \max\{L_j,0\}\\
E_j & = \max\{-L_j,0\}
\end{align*}
常见基本$\gamma$域有:\\[3pt]
\begin{supertabular}{ll}
$\sum w_jC_j$ & 加权订单完成时刻总和 \\
$\sum w_jT_j$ & 加权订单延迟时间总和 \\
\end{supertabular}

\subsection{基本假设}
在模型建立之前,需要进行一些基本假设。
\begin{compactenum}
\item 整数变量假设
\suspend{compactenum}

模型中所涉及的所有变量,如订单处理时间$p_j$,切换准备时间$s_j$等,皆在整数范围内考虑(除了后面的决策参数$\lambda_1, \lambda_2$),便于后面将问题离散化。
\resume{compactenum}
\item 数量有限假设
\suspend{compactenum}

假设研究对象所设计的订单数量$n$、订单包含的作业数量$n_j$、订单涉及的处理单元数量$m_j$以及流水线数量$m$皆有限。
\resume{compactenum}
\item 无差别假设
\suspend{compactenum}

对于订单来说,一般都是由一定批量的作业组成的,所以可以假设其中包含的各作业无差别,即每项作业的处理时间、工艺顺序皆相同。由于各流水线上无固定设备,在安排装配生产时,都是现场安排流程对应的工位,所以可以假设流水线之间无差别,类似于同速并行机环境,即同一订单在不同流水线上的切换准备时间、处理时间都一样。
\resume{compactenum}
\item 无插单生产假设
\suspend{compactenum}

模型1 考虑订单到达比较稳定的情况,插单的出现很少,可以忽略其影响,所以可以假设,没有插单发生。假设所有订单在系统时刻$t = 0$时下达进入流水线系统,皆可开始安排生产处理,即$\forall r_j =0$,并且之后没有新的订单进入系统。
\resume{compactenum}
\item 不可中断假设
\suspend{compactenum}

订单在流水线上装配处理过程中,需要其中所有的作业装配完成才可以离开流水线系统,假设生产过程中没有人为或机器原因产生中断。同时,由于该厂总装厂的产品体积较小,每个工位都有足够的空间存放在制品,不会因此使生产中断。
\resume{compactenum}
\item 无相关假设
\suspend{compactenum}

按队列生产时,处理不同订单需要考虑切换准备时间,由于流水线上的工位都是订单到达后再根据工艺布置的,所以可以假设切换准备时间只和订单本身有关,和其前续后继订单没有相关性,并且各订单所需的切换准备时间事先已知。此外,各流水线切换准备时间不算入停线等待。
\resume{compactenum}
\item 惩罚假设
\end{compactenum}

每个订单都有一定的批量,若订单延迟(提早),虽然对于这个订单来说尚未完成,但该订单中已完成的作业是可以交付的,所以假设订单相应惩罚只和其所延迟(提早)的作业数量有关。
\subsection{目标函数}

本课题期望通过合理调度,达到提高流水线利用率、减少浪费、提高按时交货率、缩短制造期等目标,这些目标有着内在联系,比如最小化完成时刻在一定程度上相当于最大化处理单元利用率,进一步可以暗示最大化流水线利用率。

对于模型1 ,考虑满足工期和提高流水线利用率,其中满足工期为主要目标,可用加权延迟时间和$\sum wt_jT_j$,流水线利用率为次要目标,可用加权完成总时刻和$\sum wc_jC_j$。其中$wt_j, wc_j$分别为订单$j$的延迟和完工权重。两目标的重要程度可以体现在目标权重系数$\lambda_1, \lambda_2$上。

进一步分析该问题,可以发现由不可中断假设和无差别假设,任一订单的总装配生产时间是确定的,并且与其被加工的流水线无关,所以可将各装配线看作是并行同速机环境,每条流水线看作一个可以处理所有订单的机器。该问题可以记为:$Pm \mid s_j\mid\lambda_t\sum wt_jT_j + \lambda_c\sum wc_jC_j$\ ,那么该目标函数可表示为:

\begin{equation}
\min\quad \lambda_t\sum_{j = 1}^n wt_jT_j +\lambda_c\sum_{j=1}^n wc_jC_j
\label{equ:primeobj}
\end{equation}

目标权重系数的确定涉及到管理者的决策,不同的比例对应于不同的生产。根据\eqref{equ:primeobj}的特性,并从流水线角度来考虑,可以改写成如下等价形式:
\begin{equation}
\min\quad \lambda_t\sum_{l=1}^m\sum_{k=1}^{|S_l|} wt_{l_k}T_{l_k} + \lambda_c\sum_{l=1}^m\sum_{k=1}^{|S_l|}wc_{l_k}C_{l_k}
\label{equ:objmain}
\end{equation}
\subsection{约束条件}
约束条件可以从订单间的关系中寻找,
并行机环境中,可以根据流水线来考虑订单。记订单$l_k$的处理时间为$p'_{l_k}$,由于其准备时间为定值$s_{l_k}$,可以将其并入订单处理时间来简化问题而不影响结果,并记订单$l_k$的整合切换准备处理时间为$p_{l_k}$,那么该模型1 的主要约束如下:
\begin{numcases}{}
\sum_{l=1}^m |S_l| = n\label{equ:basicst1}\\
\bigcup_{l=1}^m \overline{S_l} = N\label{equ:basicst2}\\
\sum_{l=1}^m\sum_{k=1}^{|S_l|} wt_{l_k}= 1\\
\sum_{l=1}^m\sum_{k=1}^{|S_l|} wc_{l_k}= 1\\
\lambda_c + \lambda_t = 1\\
C_{l_1} = p_{l_1} & $l = 1,2,...,m$\label{equ:basicst3}\\
C_{l_k} = C_{l_{k-1}} + p_{l_k} & $k = 2,3,...,|S_l|, l = 1,2,...,m$\label{equ:basicst4}\\
p_{l_k} = p'_{l_k} + s_{l_k} & $k = 1,2,...,|S_l|, l = 1,2,...,m$\label{equ:basicst5}\\
T_{l_k} = \max\{0, C_{l_k} - d_{l_k}\} & $k = 1,2,...,|S_l|, l = 1,2,...,m$\label{equ:basicst6}\\
p'_{l_k}, s_{l_k}, d_{l_k}, wt_{l_k}, \lambda_t, \lambda_c\ge 0 & $k = 1,2,...,|S_l|, l = 1,2,...,m$\label{equ:basicst7}
\end{numcases}
\eqref{equ:basicst1}表示所有流水线安排的订单数量总和为需要安排调度的订单数量总和,\eqref{equ:basicst2}为\eqref{equ:basicst1}的集合表达形式,\eqref{equ:basicst3} -- (\ref{equ:basicst4})计算各订单的完成时刻,连续的订单之间没有停顿,\eqref{equ:basicst5}计算了整合订单处理时间,\eqref{equ:basicst6}为订单的延迟定义,\eqref{equ:basicst7}为变量的非负约束。

\section{考虑插单的多品种多装配线轮番装配调度优化模型}
该模型的设计是考虑到了订单到达不稳定的情况,即会产生许多的插入订单,考虑订单插单的情况所建的模型简称模型2。在系统在开始运行后,订单陆续进入系统,其进入系统的时刻为该订单最早可被处理时刻。也就是说各订单并不是在系统时刻$t=0$时刻皆可开始进行处理,订单可以开始处理需要同时满足两个条件:
\begin{inparaenum}
\renewcommand{\theenumi}{\protect\setcounter{local}{171 + \the\value{enumi}}\protect\ding{\value{local}}}
\renewcommand{\labelenumi}{\theenumi}
\item 有流水线处于闲置状态,
\item 系统时刻大于等于订单最早可开始时刻$t \ge r_j$。
\end{inparaenum}
但若有流水线闲置,且接下来要对这个订单进行加工处理,那么此时可以先开始该订单的切换准备,以节少流水线产能浪费。

模型2 对工期的要求进一步提高,不仅要求订单不产生延迟,更对订单的提早完成作出相应惩罚,这是调度研究的最新热点,符合准时化生产的要求,可以很好体现一个企业的管理水平。此外,模型2 更为深入挖掘流水线的性能,考虑了各流水线的利用率和流水线的整体均衡性。
这些要求都使得模型2 更为接近实际情况。
\subsection{相关符号及说明}
模型2 建立在模型1 的基础上,由于有新的因素加入考虑范畴,需要补充或修改符号定义如下:

\begin{supertabular}{ll}
$C_l$ & 流水线$l$的制造期,其值为$C_{l_{|S_l|}}$ \\
$f_j$ & 订单$j$开始处理前,处理流水线的闲置时间\\
$Rb$ & 流水线均衡率 \\
$Ru_l$ & 流水线$l$的利用率\\ 
\end{supertabular}

\subsection{相关假设}
模型2 比模型1 多考虑了插单的情况,所以需要增加和改变模型1 的相关假设。
\begin{compactenum}
\item 插单假设
\suspend{compactenum}

模型2 考虑了插单情况,所以模型1 的无插单假设不在适用,而插单的情况相当于各订单陆续进入系统,即$\exists r_{l_k} >0$。
\resume{compactenum}
\item 惩罚一致假设
\suspend{compactenum}

由于对订单的交付准时要求有所提高,故可以将订单的延迟和提早做等价的惩罚,即$w_e = w_t$,这样一来可以便于后续目标函数的改写。
\resume{compactenum}
\item 订单最早可处理时刻假设
\end{compactenum}

考虑插单的情况会导致流水线的空闲等待,而订单开始前都需要经过切换准备,假设订单的切换准备可以在其最早可处理时刻$r_j$之前准备,而装配生产必须在$r_j$之后方可开始。这样假设可以提高流水线的利用率,也是十分合理的。
\subsection{目标函数}
考虑订单陆续到达时,更为注重订单的按时交付,同时也关注流水线的生产均衡性。生产均衡性指的是流水线的使用均衡,不要出现某条流水线一直繁忙而有些流水线空闲居多,导致负荷不均衡,损失产能。流水线均衡率定义如下:

\[
Rb = \frac{\sum_{l=1}^m C_l}{\displaystyle m\times \max_{1 \le l \le m} \{C_l\}}
\]

模型1 中,各订单没有可处理时刻的限制,各流水线除了切换准备,其余时间都在处理订单,研究流水线利用率是没有意义的。而在模型2 中,流水线上订单间的空闲等待将会出现,其中切换准备同样不计入空闲。为了提高流水线利用率,需要对其进行定义。假设订单$j,k$为同条流水线上的连续处理订单,且订单$j$先于订单$k$处理,需要考虑一下$3$种情况:

\begin{asparaenum}
\item $C_j \ge r_k$
\suspend{asparaenum}

这种情况表明,订单$k$进入系统的时候,其前续作业$j$仍然在处理当中,那么显然处理他们的流水线是不存在闲置的,即$f_k = 0$。
\resume{asparaenum}
\item $C_j < r_k, C_j\ge r_k - s_k$
\suspend{asparaenum}

这种情下,虽然订单$k$进入系统可开始处理的时间在订单$j$之后,然而由假设生产计划部门可以提前安排该订单的准备,所以在前续订单$j$处理完成后,先进行切换准备。然而该切换准备完成时,订单已处于可加工状态,所以整体来说,该流水线仍然没有闲置,即$f_k = 0$。
\resume{asparaenum}
\item $C_j < r_k - s_k$
\end{asparaenum}

这种情况较上种情况不同,订单之间的间隔时间大于后继订单的切换准备时间,那么就会使得该流水线出现闲置等待,此时$f_k = r_k - s_k -C_j$。

此外,需要人为定义首项订单的闲置$f_{l_1} = \max\{r_{l_k} - s_{l_k}, 0\}, (l = 1,2,...,m)$,综上,可以定义订单$l_k$开始处理前的流水线闲置:

\begin{subnumcases}{f_{l_k} = }
\max\{r_{l_k} - s_{l_k}, 0\} & $k = 1$\notag\\
\max\{r_{l_k} - s_{l_k}- C_{l_{k-1}}, 0\}& $k\ge 2$\notag
\end{subnumcases}

由此,可以定义流水线$l$的利用率:
\[
Ru_l = 1 - \frac{\sum_{k=1}^{|S_l|}f_{l_k}}{C_l}
\]
表示该流水线在工期内处于生产处理状态的比重。
和流水线均衡率不同,流水线利用率针对每条流水线本身,而流水线均衡率则是从全局的角度。

为了达到准时交货的目标,需要将主要目标定为加权延迟时间和与加权提早时间和的和,并有假设,这两个指标共用一个权重,即:

\[
\min \quad \sum_{j = 1}^n w_j(T_j + E_j)
\]
根据这个式子的特点,可将其改写成等价形式:

\[
\min \quad \sum_{j = 1}^n w_j|L_j|
\]
由于主要目标和工期的关联很大,容易受其牵制而安排出利用率不高的调度,故需将产线利用率考虑其中,并从流水线的角度来考虑,则主要需改写成\eqref{equ:insertmainobj}。
\begin{equation}
\min \quad \sum_{l = 1}^m\frac{\sum_{k=1}^{|S_l|} w_{l_k}|L_{l_k}|}{Ru_l}\label{equ:insertmainobj}
\end{equation}

此外,需要考虑流水线的平衡性,可以综合入次要目标加权完成总和当中,其形式可以有很多种,为了体现提高流水线平衡性的重要,可以这样设计:
\begin{equation}
\min \quad e^{-Rb}\sum_{l=1}^m\sum_{k=1}^{|S_l|}wc_{l_k}C_{l_k}
\label{equ:insertsecondobj}
\end{equation}

与模型1 类似,主要和次要目标可以通过目标权重系数$\lambda$上。结合\eqref{equ:insertmainobj}与(\ref{equ:insertsecondobj})得到本模型的目标函数:
\begin{equation}
\min \quad \lambda_1\sum_{l = 1}^m\frac{\sum_{k=1}^{|S_l|}w_{l_k}|L_{l_k}|}{Ru_l} + \lambda_2 e^{- Rb}\sum_{l=1}^m\sum_{k=1}^{|S_l|}wc_{l_k}C_{l_k}
\label{equ:insertobj}
\end{equation}

\subsection{约束条件}
模型2 以模型1 为基础,需要多考虑订单进入时刻,所以要对基本约束进行修改和增加:
\begin{numcases}{}
\sum_{l=1}^m |S_l| = n\label{equ:insertst1}\\
\bigcup_{l=1}^m \overline{S_l} = N\label{equ:insertst2}\\
\sum_{l=1}^m\sum_{k=1}^{|S_l|} w_{l_k}= 1\label{equ:insertst3}\\
\sum_{l=1}^m\sum_{k=1}^{|S_l|} wc_{l_k}= 1\label{equ:insertst4}\\
\lambda_1 + \lambda_2 = 1\label{equ:insertst5}\\
C_{l_1} = f_{l_1} + s_{l_1} + p_{l_1}& $l = 1,2,...,m$\label{equ:insertst6}\\
C_{l_k} = C_{l_{k-1}} + f_{l_k} + s_{l_k} + p_{l_k} & $k = 2,3,...,|S_l|, l = 1,2,...,m$\label{equ:insertst7}\\
%p_{l_k} = p'_{l_k} + s_{l_k} & $k = 1,2,...,|S_l|, l = 1,2,...,m$\label{equ:insertst8}\\
\sum_{l=1}^m\sum_{k=1}^{|S_l|} r_{l_k} > 0& $k = 2,3,...,|S_l|, l = 1,2,...,m$\label{equ:insertst9}\\
L_{l_k} = C_{l_k} - d_{l_k}& $k = 1,2,...,|S_l|, l = 1,2,...,m$\label{equ:insertst10}\\
T_{l_k} = \max\{0, C_{l_k} - d_{l_k}\} & $k = 1,2,...,|S_l|, l = 1,2,...,m$\label{equ:insertst11}\\
E_{l_k} = \max\{d_{l_k} - C_{l_k}, 0\} & $k = 1,2,...,|S_l|, l = 1,2,...,m$\label{equ:insertst12}\\
s_{l_k}, d_{l_k}, w_{l_k}, wc_{l_k}, \lambda_1, \lambda_2, r_{l_k}\ge 0 & $k = 1,2,...,|S_l|, l = 1,2,...,m$\label{equ:insertst13}
\end{numcases}
\eqref{equ:insertst1}表示所有流水线安排的订单总和,\eqref{equ:insertst2}为\eqref{equ:insertst1}的集合表达形式,\eqref{equ:insertst3} -- (\ref{equ:insertst5})是权重之后为$1$,\eqref{equ:insertst6} -- (\ref{equ:insertst7})计算了各订单的完成时刻,考虑了流水线的等待时间,\eqref{equ:insertst9}确保不出现所有订单在$t = 0$时刻同时可以进行处理的情况,\eqref{equ:insertst10} -- (\ref{equ:insertst12})是订单的迟滞、延迟、提早的定义,\eqref{equ:insertst13}为变量的非负定义。

至此,$2$个多品种多装配线轮番装配调度优化模型建立完毕。
\section{小结}
本章根据上一章的现状分析和改进方案,建立了$2$个多品种多装配线轮番装配调度优化模型,模型2 比模型1 多考虑了插单的情况。以提高流水线利用率、减少浪费、提高按时交货率、缩短制造期等为目标,两个模型分别建立了相应的目标函数,模型2 更进一步定义了流水线的利用率和均衡率,并将之融入了目标函数之中。这两个模型的求解将在下一章阐述。