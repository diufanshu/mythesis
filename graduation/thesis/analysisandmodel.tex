% !Mode:: "TeX:UTF-8"
% !TEX root = ..\thesis.tex
\chapter{装配生产线分析及模型建立}
本章将对装配线生产进行分析,指出现有计划安排和调度存在的主要问题,然后针对这些问题,提出合理解决方案,并对其进行分析和数学建模。


\section{生产线分析}
课题研究对象是该汽车电子公司的总装生产线,是典型的流水车间,每条流水线负责同一主机厂的不同品种产品总装流程。装配生产根据订单批量进行安排,根据先到先服务(FCFS)规则生成任务队列。

生产线分析包括上述这些差别及其产生的影响或效果,具体描述从订单、装配到交付的流程,并分析现行调度方案的一些指标。

\subsection{现行流程描述}
现行下达订单到交付的流程如\reff{fig:orderflow}所示,销售人员接到客户订单,在确认工期后,将之送达计划部门,制定主生产计划(MRP),随后根据产品特性安排采购与厂内加工。根据任务队列与批量进行产品总装,通过质检包装后,销售人员安排运输送达至客户。
订单到达会立即安排入其对应的主机厂专用产线,当同一主机厂有多个订单同时到达时,则根据最早交货期(EDD)规则进行生产调度。

本课题的研究对象是流程中的总装调度安排,
\subsection{现行调度方案}
任务下达到车间时,需要根据订单队列及其批量进行调度安排,


现行调度方案逻辑简单,执行力强,每条装配线可分时生产不同品种的产品,而且按照厂家来安排组织生产,方便了管理。


\subsection{主要问题}
现行调度方案存在诸多问题,例如多条装配线负荷不均衡,有的任务过重,有的任务不足,负荷不均衡,一条装配线上装配的产品工艺相似性较低,导致换线时间增加,产生更长的等待。

根据现行调度情况进行问题分析,可以归纳其主要存在问题如下:
\renewcommand{\labelenumi}{(\theenumi)}
\begin{asparaenum}
\item 产线利用率低
\suspend{asparaenum}

产线利用率低主要体现在存在大量换线时间,一方面由于产线等待队列由FCFS 规则产生,同时到达的订单也只是根据EDD 规则安排,没有考虑品种装配流程间的相似性。当这种差异很大时,必然会增加换线时间,进而增加了任务间的等待。另一方面,由于产线的专用性,当有多条产线都能处理某个任务时,该任务只能在订单来自的主机厂专线上生产,闲置了可用线的生产能力。
\resume{asparaenum}
\item 生产不均衡
\suspend{asparaenum}

这里的生产均衡和混流生产中的均衡生产稍有差别,此处的不均衡现象更为宏观。来自不同主机厂的任务在不同产线上进行处理,这样一来,订单较多、较频繁的主机厂产线总是会处于繁忙状态,而订单较少的产线则呈现为停线等待居多。这种不均衡现象直接导致产能的巨大浪费,同时也间接导致了换线时间增加,因为不均衡的生产业表面订单较为集中。突破专线限制可以解决宏观不均衡问题,虽然细分到单条线上的混流生产可以进一步均衡化生产,但其需要较高的管理投入,可以考虑折衷。
\resume{asparaenum}
\item 工艺及设备和生产需求不匹配
\suspend{asparaenum}

各流水线需要有多品种加工的能力,故线上需要有相应加工工艺的设备,而时常不同主机厂所需产品可能有很高的相似性,这使得同样的设备需要在多条流水线上设置,尤其对于加工时间较短、处理批量较少的作业,过多的设备徒增成本与闲置。另一方面,加工时间厂、处理批量大的作业,台少的设备不利于生产效率,导致在制品增爹。
\resume{asparaenum}
\item 工期可控性低
\suspend{asparaenum}

工期的可控性低主要体现在应变插单的问题上,现行调度采用的是不可中断的流水作业,各流水线只能按其队列顺序进行装配作业,由于这样安排没有顾及工期的先后即订单的具体情况,导致大部分订单都需要延期交货,也存在较多订单的过早完工,增加了库存。此外,虽然插单可以较为合理安排订单加工顺序,而然会到来额外的切换时间,需要权衡考虑。
\resume{asparaenum}
\item 产线冗余度高
\end{asparaenum}

前面几大问题已经涉及到了一些浪费现象,除了这些之外,该厂制造二部共有8个装配车间,每个装配车间有7--8条总装流水线,然而由于流水线是按主机厂进行分配,存在较高的冗余度,前面提到的产线间设备类似属于其中之一。产线的冗余还包括作业及管理人员,辅助设置,场地空间,相关能源等。


\section{模型建立}
现在突破主机厂的界限 就是一条生产线可以生产多家主机厂的产品

现在根据新的目标,对所有需要生产的产品进行筛选组合,以期获得均衡的生产线利用率、减少换线时间浪费、缩短完工时间、降低生产成本

运用混流生产中的品种均衡方法可以有效解决生产平衡问题,有关混流生产的研究也十分众多,理论相对较为成熟。
然而实施均衡化生产需要有较高的管理水平,不太适用于本课题的研究对象。因此,本节将结合实际情况,权衡投入和效益,建立适合本课题研究对象的混线装配生产模型。
\subsection{方案改进}


\subsection{符号说明}

\subsection{相关假设}
按队列生产时,需要考虑切换准备时间,这个准备时间只和将要进行生产的产品有关,而和前一项作业无关。

\subsection{目标函数}

\subsection{约束条件}


\section{小结}