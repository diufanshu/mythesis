% !Mode:: "TeX:UTF-8"
% !TEX root = ..\thesis.tex
\chapter{多品种装配车间调度建模}
上一章分析了现行调度存在的问题并提出了改进方案设计,本章将进行具体改进过程的分析与建模。

\section{建模准备}

根据上一章的改进设计,对所需生产的订单进行排列组合,较为均匀地安排在各流水线上,以期获得均衡的流水线利用率、减少换线时间浪费、缩短完工时间、降低生产成本。此外,虽然实施完全均衡化生产需要有较高的管理水平和投入,但是适当的混流(即考虑投入产出关系后)可以有效提高流水线性能,需要确定混流程度。

产品订单确定其所需产品的数量(订单批量),订单中的每个产品可以看作为作业,流水线上的各装配工位或机器看作处理单元,由于本课题的研究主要内容较少涉及具体的装配工艺,故产品订单也确定了作业处理所需的处理单元的数量、种类以及顺序,流水线上的订单处理可以看作是各作业按固定顺序经过线上的处理单元进行处理。

\subsection{基本符号说明}
为了方便问题描述,需要说明基本符号如下,其中涉及的时间变量研究对象为系统时间:\\[3pt]
\begin{supertabular}{ll}
$n$ & 订单数量\\
$m$ & 流水线数量\\
$j$ & 订单标记,$j = 1,2,...,n$\\
$N$ & 所有订单集合$\{ j\mid j \in \mathbb{Z}, 1\le j \le n  \}$\\
$l$ & 流水线标记,$l = 1,2,...,m$\\
$S_l$ & 流水线$l$上的订单调度\\
$|S|$ & 调度$S$的订单数量\\
$\overline S$ & 调度$S$中的订单集合\\
$l_k$ & 调度$S_l$的第$k$项订单标记,,$k = 1,2,...,|S_l|$\\
$d_j$ & 订单$j$的交货时刻(工期)\\
$t$ & 生产系统时间\\[3pt]
\end{supertabular}

一个调度问题可以由三元组$\alpha \mid \beta \mid \gamma$表示,$\alpha$域描述单一处理单元环境,$\beta$域包含加工特征和约束的细节,$\gamma$域描述其目标\cite{pinedo}。

\subsubsection{基本$\alpha$域}
\begin{supertabular}{ll}
$Pm$ & 同速并行机\\
$Fm$ & 流水车间\\
\end{supertabular}

\subsubsection{基本$\beta$域}
\begin{supertabular}{ll}
$r_j$ & 订单$j$到达流水线系统时刻,是其最早可开始时刻\\
$s_j$ & 订单$j$开始之前所需的切换(开工)准备时间\\
\end{supertabular}

\subsubsection{基本$\gamma$域}
$\gamma$域涉及目标函数,一般调度问题需要考虑最小化目标函数,常见的目标函数为订单$j$的完成时刻$C_j$,为订单$j$离开系统的时刻。订单$j$的迟滞可以定义为:
\[
L_j = C_j - d_j
\]
进一步可定义其延迟和提前:
\begin{align*}
T_j & = \max\{L_j,0\}\\
E_j & = \max\{-L_j,0\}
\end{align*}
常见基本$\gamma$域有:\\[3pt]
\begin{supertabular}{ll}
$\sum w_jC_j$ & 加权订单完成时刻总和 \\
$\sum w_jT_j$ & 加权订单延迟时间总和 \\
\end{supertabular}

\section{多品种多装配线轮番装配调度优化模型}
\subsection{基本假设}
本课题研究所涉及的所有变量(除权重值外)皆为整数,便于后面将问题离散化。
假设订单、订单包含的作业、订单涉及的处理单元以及流水线数量有限,
订单中的每项作业均相同,相同处理单元布置在不同流水线上的处理能力不变,也就是说同订单中的作业在不同流水线的处理时刻相同。
所有订单在系统时刻$t = 0$时下达,皆可开始生产处理,即$\forall r_j =0$,并且之后没有新的订单进入系统。
处理过程中,订单不可被中断,即订单一旦开始装配处理,就要不间断地处理完整个批量。
由于该厂总装厂的产品体积较小,可以忽略流水线上处理单元间的在制品
本课题的调度研究不涉及具体的装配工艺流程,故假定不同产品的工艺区别只能体现在处理时间和设备准备时间,
按队列生产时,处理不同订单需要考虑切换准备时间,切换准备时间只和订单本身有关,并且所需准备时间事先已知。此外,各流水线除了切换准备一般不停线等待。
所有作业均完成的订单方可交付

\subsection{目标函数}
多品种多装配线轮番装配调度优化模型
本课题期望通过合理调度,达到提高流水线利用率、减少浪费、缩短制造期、提高应变能力、减少库存、降低生产成本等目标,这些目标有着内在联系,所以在设计目标函数的时候,不必把每样都列入其中,比如最小化完成时刻在一定程度上相当于最大化处理单元利用率,进一步可以暗示最大化流水线利用率。

对于基本模型,可以考虑满足工期和提高流水线利用率,其中满足工期为主要目标,可用加权延迟时间和$\sum wt_jT_j$,流水线利用率为次要目标,可用加权完成总时刻和$\sum wc_jC_j$。其中$wt_j, wc_j$分别为订单$j$的延迟和完工权重。两目标的重要程度可以体现在目标权重系数$\lambda$上。

进一步分析该问题,可以发现由给出的基本假设订单不可中断,那么每个订单的总装配生产时间是确定的,并且与其被加工的流水线无关,所以可将各装配线看作是并行同速机环境,每条流水线看作一个可以处理所有订单的机器。该问题可以记为:$Pm \mid s_j\mid\lambda_t\sum wt_jT_j + \lambda_c\sum wc_jC_j$\ ,那么该目标函数可表示为:

\begin{equation}
\min\quad \lambda_t\sum_{j = 1}^n wt_jT_j +\lambda_c\sum_{j=1}^n wc_jC_j
\label{equ:primeobj}
\end{equation}

目标权重系数的确定涉及到管理者的决策,不同的比例对应于不同的生产,可以进行帕累托分析。根据\eqref{equ:primeobj}的特性,并从流水线角度来考虑,可以改写成如下等价形式:
\begin{equation}
\min\quad \lambda_t\sum_{l=1}^m\sum_{k=1}^{|S_l|} wt_{l_k}T_{l_k} + \lambda_c\sum_{l=1}^m\sum_{k=1}^{|S_l|}wc_{l_k}C_{l_k}
\label{equ:objmain}
\end{equation}
\subsubsection{约束条件}
约束条件需要从订单间的关系中寻找,
并行机环境中,可以根据流水线来考虑订单。记订单$l_k$的处理时间为$p'_{l_k}$,由于其准备时间为定值$s_{l_k}$,可以将其并入订单处理时间来简化问题而不影响结果,并记订单$l_k$的整合切换准备处理时间为$p_{l_k}$,那么该基本模型的主要约束如下:
\begin{numcases}{}
\sum_{l=1}^m |S_l| = n\label{equ:basicst1}\\
\bigcup_{l=1}^m \overline{S_l} = N\label{equ:basicst2}\\
\sum_{l=1}^m\sum_{k=1}^{|S_l|} wt_{l_k}= 1\\
\sum_{l=1}^m\sum_{k=1}^{|S_l|} wc_{l_k}= 1\\
\lambda_c + \lambda_t = 1\\
C_{l_1} = p_{l_1} & $l = 1,2,...,m$\label{equ:basicst3}\\
C_{l_k} = C_{l_{k-1}} + p_{l_k} & $k = 2,3,...,|S_l|, l = 1,2,...,m$\label{equ:basicst4}\\
p_{l_k} = p'_{l_k} + s_{l_k} & $k = 1,2,...,|S_l|, l = 1,2,...,m$\label{equ:basicst5}\\
T_{l_k} = \max\{0, C_{l_k} - d_{l_k}\} & $k = 1,2,...,|S_l|, l = 1,2,...,m$\label{equ:basicst6}\\
p'_{l_k}, s_{l_k}, d_{l_k}, wt_{l_k}, \lambda_t, \lambda_c\ge 0 & $k = 1,2,...,|S_l|, l = 1,2,...,m$\label{equ:basicst7}
\end{numcases}
式中45464

式中45464

式中45464

式中45464
\section{考虑插单的多品种多装配线轮番装配调度优化模型}
陆续模型考虑系统在开始运行后,订单陆续进入系统,其进入系统的时刻为该订单最早可被处理时刻。也就是说各订单并不是在系统时间$t=0$时刻皆可开始进行处理,但若有流水线闲置,且接下来要对这个订单进行加工处理,那么此时可以先开始该订单的切换准备,以节少流水线产能浪费。陆续模型对工期的要求进一步提高,不仅要求订单不产生延迟,更对订单的提早完成作出相应惩罚,这是调度研究的最新热点,符合准时化生产的要求,可以很好体现一个企业的管理水平。此外,陆续模型更为深入挖掘流水线的性能,考虑了各流水线的利用率和流水线的整体均衡性。
这些要求都使得陆续模型更为接近实际情况。
\subsection{相关符号及说明}
陆续模型建立在基本模型的基础上,由于有新的因素加入考虑范畴,需要补充或修改符号定义如下:

\begin{supertabular}{ll}
$C_l$ & 流水线$l$的制造期,其值为$C_{l_{|S_l|}}$ \\
$f_j$ & 订单$j$开始处理前,处理流水线的闲置时间\\
$Rb$ & 流水线均衡率 \\
$Ru_l$ & 流水线$l$的利用率\\ 
\end{supertabular}

\subsection{相关假设}
根据订单到达先后动态安排调度的难度较大,而且也难以实施应用,故可以假设所有订单在$t=0$时皆已知,陆续到达的订单相当于指定了某订单的最早可开始时间,即$\exists r_{l_k} >0$。
系统时间由订单情况而定,假定系统从开始换线为时间起始,即要保证$\min(r_{l_k}) = 0$,需要进一步处理,记各订单进入流水线时间为$r'_{l_k}$,则变换后的进入系统时间为$r_{l_k} = r'_{l_k} - \min(r'_{l_k})$。
需假设所有订单不可被中断
延迟的订单和提早交货的订单的单位时间惩罚量一致

\subsection{目标函数}
考虑订单陆续到达时,更为注重订单的按时交付,同时也关注流水线的生产均衡性。生产均衡性指的是流水线的使用均衡,不要出现某条流水线一直繁忙而有些流水线空闲居多,导致负荷不均衡,损失产能。流水线均衡率定义如下:

\[
Rb = \frac{\sum_{l=1}^m C_l}{\displaystyle m\times \max_{1 \le l \le m} \{C_l\}}
\]

基本模型中,各订单没有可处理时刻的限制,各流水线除了切换准备,其余时间都在处理订单,研究流水线利用率是没有意义的。而在陆续模型中,流水线上订单间的空闲等待将会出现,其中切换准备同样不计入空闲。为了提高流水线利用率,需要对其进行定义。假设订单$j,k$为同条流水线上的连续处理订单,且订单$j$先于订单$k$处理,需要考虑一下$3$种情况:

\begin{asparaenum}
\item $C_j \ge r_k$
\suspend{asparaenum}

这种情况表明,订单$k$进入系统的时候,其前续作业$j$仍然在处理当中,那么显然处理他们的流水线是不存在闲置的,即$f_k = 0$。
\resume{asparaenum}
\item $C_j < r_k, C_j\ge r_k - s_k$
\suspend{asparaenum}

这种情下,虽然订单$k$进入系统可开始处理的时间在订单$j$之后,然而由假设生产计划部门可以提前安排该订单的准备,所以在前续订单$j$处理完成后,先进行切换准备。然而该切换准备完成时,订单已处于可加工状态,所以整体来说,该流水线仍然没有闲置,即$f_k = 0$。
\resume{asparaenum}
\item $C_j < r_k - s_k$
\end{asparaenum}

这种情况较上种情况不同,订单之间的间隔时间大于后继订单的切换准备时间,那么就会使得该流水线出现闲置等待,此时$f_k = r_k - s_k -C_j$。

此外,需要人为定义首项订单的闲置$f_{l_1} = \max\{r_{l_k} - s_{l_k}, 0\}, (l = 1,2,...,m)$,综上,可以定义订单$l_k$开始处理前的流水线闲置:

\begin{subnumcases}{f_{l_k} = }
\max\{r_{l_k} - s_{l_k}, 0\} & $k = 1$\notag\\
\max\{r_{l_k} - s_{l_k}- C_{l_{k-1}}, 0\}& $k\ge 2$\notag
\end{subnumcases}

由此,可以定义流水线$l$的利用率:
\[
Ru_l = 1 - \frac{\sum_{k=1}^{|S_l|}f_{l_k}}{C_l}
\]
表示该流水线在工期内处于生产处理状态的比重。
和流水线均衡率不同,流水线利用率针对每条流水线本身,而流水线均衡率则是从全局的角度。

为了达到准时交货的目标,需要将主要目标定为加权延迟时间和与加权提早时间和的和,并有假设,这两个指标共用一个权重,即:

\[
\min \quad \sum_{j = 1}^n w_j(T_j + E_j)
\]
根据这个式子的特点,可将其改写成等价形式:

\[
\min \quad \sum_{j = 1}^n \sqrt{w_j^2L_j^2}
\]
由于主要目标和工期的关联很大,容易受其牵制而安排出利用率不高的调度,故需将产线利用率考虑其中,并从流水线的角度来考虑,则主要需改写成\eqref{equ:insertmainobj}。
\begin{equation}
\min \quad \sum_{l = 1}^m\frac{\sum_{k=1}^{|S_l|}\sqrt{w_{l_k}^2L_{l_k}^2}}{Ru_l}\label{equ:insertmainobj}
\end{equation}

此外,需要考虑流水线的平衡性,可以综合入次要目标加权完成总和当中,其形式可以有很多种,为了体现提高流水线平衡性的重要,可以这样设计:
\begin{equation}
\min \quad \exp\left(-\frac{Rb}{\sigma}\right)\sum_{l=1}^m\sum_{k=1}^{|S_l|}wc_{l_k}C_{l_k}
\label{equ:insertsecondobj}
\end{equation}

式中$\sigma$是比例参数,其值越大,对目标值影响越大,表明均衡性越为重要,需要根据生产管理人员的决策来确定。与基本模型类似,主要和次要目标可以通过目标权重系数$\lambda$上,然后进行帕累托分析。结合\eqref{equ:insertmainobj}与(\ref{equ:insertsecondobj})得到本模型的目标函数:
\begin{equation}
\min \quad \lambda_1\sum_{l = 1}^m\frac{\sum_{k=1}^{|S_l|}\sqrt{w_{l_k}^2L_{l_k}^2}}{Ru_l} + \lambda_2 \exp\left(-\frac{Rb}{\sigma}\right)\sum_{l=1}^m\sum_{k=1}^{|S_l|}wc_{l_k}C_{l_k}
\label{equ:insertobj}
\end{equation}

\subsection{约束条件}
陆续模型是以基本模型为基础,需要多考虑订单进入时刻,所以要对基本约束进行修改和增加:
\begin{figure}[h]
\begin{numcases}{}
\sum_{l=1}^m |S_l| = n\label{equ:insertst1}\\
\bigcup_{l=1}^m \overline{S_l} = N\label{equ:insertst2}\\
\sum_{l=1}^m\sum_{k=1}^{|S_l|} w_{l_k}= 1\label{equ:insertst3}\\
\sum_{l=1}^m\sum_{k=1}^{|S_l|} wc_{l_k}= 1\label{equ:insertst4}\\
\lambda_1 + \lambda_2 = 1\label{equ:insertst5}\\
C_{l_1} = f_{l_1} + s_{l_1} + p_{l_1}& $l = 1,2,...,m$\label{equ:insertst6}\\
C_{l_k} = C_{l_{k-1}} + f_{l_k} + s_{l_k} + p_{l_k} & $k = 2,3,...,|S_l|, l = 1,2,...,m$\label{equ:insertst7}\\
p_{l_k} = p'_{l_k} + s_{l_k} & $k = 1,2,...,|S_l|, l = 1,2,...,m$\label{equ:insertst8}\\
\sum_{l=1}^m\sum_{k=1}^{|S_l|} r_{l_k} > 0& $k = 2,3,...,|S_l|, l = 1,2,...,m$\label{equ:insertst9}\\
L_{l_k} = C_{l_k} - d{l_k}& $k = 1,2,...,|S_l|, l = 1,2,...,m$\label{equ:insertst10}\\
T_{l_k} = \max\{0, C_{l_k} - d_{l_k}\} & $k = 1,2,...,|S_l|, l = 1,2,...,m$\label{equ:insertst11}\\
E_{l_k} = \max\{d_{l_k} - C_{l_k}, 0\} & $k = 1,2,...,|S_l|, l = 1,2,...,m$\label{equ:insertst12}\\
s_{l_k}, d_{l_k}, w_{l_k}, wc_{l_k}, \lambda_1, \lambda_2, r_{l_k}\ge 0, \sigma >0 & $k = 1,2,...,|S_l|, l = 1,2,...,m$\label{equ:insertst13}
\end{numcases}
\end{figure}


\section{小结}