% !Mode:: "TeX:UTF-8"
% !TEX root = ..\thesis.tex
\chapter{计算实验及算法评估}
本章将对上一章的算法进行评估,通过设计相关计算实验,建立评价体系,具体评估各算法的效果及其适用规模。然后根据实验结果,为研究对象制定合适的调度方案。

\section{实验设计}
实验设计分为两个主要部分,一个是装配生产信息的生成,包括订单数量、流水线数量、各订单切换准备时间、个订单进入系统时间及其所含作业的信息;另一个是相关参数的确定,包括迭代次数、禁忌列表长度的等。
\subsection{装配信息生成}

\subsection{相关参数确定}
到达时间:Poisson分布
处理时间:负指数分布(指数分布)、艾尔朗分布

由\reft{tab:2jobshopinfo},流水线的数量可以是$5,6,7$,产品品种的范围是$29\~ 777$,所以合理的作业数量为$30,50,100,200,500,1000$
,作业处理时间单位为转换过的 $t.u.$,由于每个订单的作业批量都比较大,所以订单处理时间单位为$500 t.u.$
产品特点是体积小,而每个订单所含的产品数量,即需要考虑的作业数量大,一般在$1000\~ 2000$左右

\section{算法评价体系}



\section{参数确定}
\section{结果及评估}


\section{小结}