% !Mode:: "TeX:UTF-8"
% !TEX root = ..\thesis.tex
\chapter{计算实验及算法评估}
本章将对上一章的算法进行评估,通过设计相关计算实验,建立评价体系,具体评估各算法的效果及其适用规模。然后根据实验结果,为研究对象制定合适的调度方案。
\section{算法评价体系}

到达时间:Poisson分布
处理时间:负指数分布(指数分布)、艾尔朗分布

由\reft{tab:2jobshopinfo},流水线的数量可以是$5,6,7$,产品品种的范围是$29\~ 777$,所以合理的作业数量为$30,50,100,200,500,1000$
,作业处理时间单位为转换过的 $t.u.$,由于每个订单的作业批量都比较大,所以订单处理时间单位为$500 t.u.$
产品特点是体积小,而每个订单所含的产品数量,即需要考虑的作业数量大,一般在$1000\~ 2000$左右
\section{实验设计}
\lstset{	basicstyle = \small\ttfamily,
	keywordstyle = \color{blue}\bfseries,
	stringstyle = \color{red},
	emph = {solve},
	emphstyle = \color{Green}\bfseries,
	commentstyle = \color{CadetBlue}
	}
\begin{lstlisting}[language = Python]
def solve(input_data):
	data = input_data.split('\n')		# load data
	n = len(data)		
	N = range(n)



import sys
if __name__ == '__main__':
	if len(sys.argv) > 1:
		file_location = sys.argv[1].strip()
		output = sys.argv[2].strip()
		input_data_file = open(file_location, 'r')
		input_data = ''.join(input_data_file.readlines())
		input_data_file.close()
		solve(input_data)


\end{lstlisting}

\section{结果及评估}


\section{小结}