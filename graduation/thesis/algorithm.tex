% !Mode:: "TeX:UTF-8"
% !TEX root = ..\thesis.tex
\chapter{算法设计}
上一章构建了4个多品种产品调度的数学模型,需要对其检验,而这些模型求解是NP--Hard问题,需要设计相应算法来求解。在求解模型前,需要对订单(子单)处理时间计算,该问题也需通过算法设计求解。
\section{处理时间函数}
\eqref{equ:processing}为订单(子单)处理时间函数,由订单特点和作业数量确定,需要补充符号说明如下:\\[3pt]
\begin{supertabular}{ll}
$k$ & 作业标记,$k = 1,2,...,n_j$\\
$m_j$ & 订单$j$的处理单元数量\\
$i$ & 订单$j$的处理单元标记,$i = 1,2,...,m_j$\\
$q_{k,i}$ & 作业$k$在处理单元$i$上的处理时间\\
$c_{k,i}$ &作业$k$的在处理单元$i$上的完成时刻\\
$c_{j,\max}$ & 订单$j$中所有作业的制造期,即$\max(c_k)$\\ 
\end{supertabular}\\[3pt]
其中不同作业在同处理单元上的处理时间相同,可以简记为$q_i$。

由于订单和子单仅有作业数量的差别,故考虑订单的情况即可推广到子单。由于,产品体积小,可以假设处理单元间的缓冲空间足够用。订单在某条流水线上的作业可以看作是流水车间环境,没有换线时间,所以该问题可以记为:$Fm\mid \mid C_{\max}$,显而易见的是 $c_{j,\max} = c_{n_j,m_j}$。该问题的数学模型如下:
\begin{gather}
\min c_{n_j, m_j}\\[-2pt]
\text{s.t.}\notag
\end{gather}
\begin{numcases}{}
c_{1,1} = q_{1,1}\\
c_{1,i} = \sum_{i=1}^{m_j} q_{1,i}\\
c_{k,1} = c_{k-1,1} + q_{k,1} & $k = 2,3,...,n_j$\\
c_{k,i} = \max(c_{k-1,i} ,c_{k,i-1}) &$k = 2,3,...,n_j, i = 2,3,...,m_j$\\
q_{k,i}  = q_i & $k = 1,2,...,n_j, i = 1,2,...,m_j$
\end{numcases}

该问题可以用有向图的关键路径进行求解,如\ref{fig:directedgraph}所示,每个节点内表示作业的处理时间,横向表示作业处理顺序,纵向表示不同的处理单元,由左上角开始,按照有向弧的方向进入节点,计算出右下节点的时间即为订单处理时间。
\begin{figure}[h]
\newcommand{\process}[1]{*++=[o][F]{\makebox[2em]{$#1$}}}
\begin{equation*}
\xymatrix{
\process{q_{1,1}} \ar[r] \ar[d] & \process{q_{1,2}} \ar[d] \ar[r] & \cdots\ar[r] & \process{q_{1,m_j}} \ar[d]\\
\process{q_{2,1}} \ar[r]\ar[d] & \process{q_{2,2}} \ar[d]\ar[r] & \cdots \ar[r] \ar[d]& \process{q_{2,m_j}} \ar[d] \\
\vdots\ar[d] & \vdots \ar[d]\ar[r] & \process{q_{k,i}}\ar[r]\ar[d] &\vdots \ar[d]\\
\process{q_{n_j,1}} \ar[r] & \process{q_{n_j,2}} \ar[r] & \cdots\ar[r] & \process{q_{n_j,m_j}} 
}
\end{equation*}
\caption{订单的作业处理有向图\label{fig:directedgraph}}
\end{figure}

\section{基本模型}
\subsection{初始解}

\subsection{搜索}


\section{小结}