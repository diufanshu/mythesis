% !Mode:: "TeX:UTF-8"
% !TEX root = ..\thesis.tex
\chapter{算法设计}
上一章构建了4个多品种产品调度的数学模型,需要对其检验,而这些模型求解是NP--Hard问题,需要设计相应算法来求解。在求解模型前,需要对订单(子单)处理时间计算,该问题也需通过算法设计求解。
\section{处理时间函数}
\eqref{equ:processing}为订单(子单)处理时间函数,由订单特点和作业数量确定,需要补充符号说明如下:\\[3pt]
\begin{supertabular}{ll}
$k$ & 作业标记,$k = 1,2,...,n_j$\\
$m_j$ & 订单$j$的处理单元数量\\
$i$ & 订单$j$的处理单元标记,$i = 1,2,...,m_j$\\
$q_{k,i}$ & 作业$k$在处理单元$i$上的处理时间\\
$c_{k,i}$ &作业$k$的在处理单元$i$上的完成时刻\\
$c_{j,\max}$ & 订单$j$中所有作业的制造期,即$\max(c_k)$\\
\end{supertabular}\\[3pt]
其中不同作业在同处理单元上的处理时间相同,可以简记为$q_i$。

由于订单和子单仅有作业数量的差别,故考虑订单的情况即可推广到子单。由于,产品体积小,可以假设处理单元间的缓冲空间足够用。订单在某条流水线上的作业可以看作是流水车间环境,没有换线时间,所以该问题可以记为:$Fm\mid \mid C_{\max}$,显而易见的是 $c_{j,\max} = c_{n_j,m_j}$。该问题的数学模型如下:
\begin{gather}
\min c_{n_j, m_j}\\[-2pt]
\text{s.t.}\notag
\end{gather}
\begin{numcases}{}
c_{1,1} = q_{1,1}\label{equ:processtime1}\\
c_{1,i} = \sum_{i=1}^{m_j} q_{1,i}\label{equ:processtime2}\\
c_{k,1} = c_{k-1,1} + q_{k,1} & $k = 2,3,...,n_j$\\
c_{k,i} = \max(c_{k-1,i} ,c_{k,i-1}) &$k = 2,3,...,n_j, i = 2,3,...,m_j$\\
q_{k,i}  = q_i & $k = 1,2,...,n_j, i = 1,2,...,m_j$
\end{numcases}

该问题可以用有向图来表示,其关键路径即为所求解的计算历程。如\ref{fig:directedgraph}所示,每个节点内表示作业的处理时间,横向表示作业处理顺序,纵向表示不同的处理单元,由左上角开始,按照有向弧的方向进入节点,计算出右下节点的时间即为订单处理时间。
其中,\eqref{equ:processtime1}和(\ref{equ:processtime2})可以化简得到这个等式:$c_{k,1} = k\cdot q_1,(k = 1,2,...,n)$,便于下面算法的初始化。
\begin{figure}[h]
\newcommand{\process}[1]{*++=[o][F]{\makebox[2em]{$#1$}}}
\begin{equation*}
\xymatrix{
\process{q_{1,1}} \ar[r] \ar[d] & \process{q_{1,2}} \ar[d] \ar[r] & \cdots\ar[r] & \process{q_{1,m_j}} \ar[d]\\
\process{q_{2,1}} \ar[r]\ar[d] & \process{q_{2,2}} \ar[d]\ar[r] & \cdots \ar[r] \ar[d]& \process{q_{2,m_j}} \ar[d] \\
\vdots\ar[d] & \vdots \ar[d]\ar[r] & \process{q_{k,i}}\ar[r]\ar[d] &\vdots \ar[d]\\
\process{q_{n_j,1}} \ar[r] & \process{q_{n_j,2}} \ar[r] & \cdots\ar[r] & \process{q_{n_j,m_j}}
}
\end{equation*}
\caption{订单的作业处理有向图\label{fig:directedgraph}}
\end{figure}

\newcounter{algor}%\newcounter{exam}
\theoremheaderfont{\heiti}
\newtheorem{algori}[algor]{算法}%\newtheorem{exam}[exam]{示例}
\begin{algori}
处理时间函数生成算法
\end{algori}

\begin{asparaenum}
\renewcommand{\labelenumi}{\bf Step\theenumi~}
\item 初始化,输入作业数量$n$和所需处理单元数量$m$,然后输入各单元处理时间$q_1,q_2,...,q_m$;
\item 计算$c_{k,1} = k\cdot q_1,(k = 1,2,...,n)$,记$i = 1$;
\item $i = i + 1, c_{1,i} = c_{1,i-1} + q_i$,记$k = 1$;
\item $k = k + 1, c_{k,i} = \max(c_{k,i-1}, c_{k-1,i}) + q_i$;
\item 如果$k<n$,执行\Step{3},否则执行\Step{5};
\item 如果$i<m$,执行\Step{2},否则结束算法。
\end{asparaenum}

\section{分派规则}
前文已提到的EDD规则和FCFS规则是常见的调度分派规则,在具体调度作业的时候,通常会先根据分派规则进行安排,所以分派规则便是作业安排的初始策略,然后再局部调整调度方案以进一步优化。一个周详的规则可能得到最优调度的初始解,显然简化了局部调整过程,然而极可能会需要巨大的思考空间和时间(比如枚举)。同样,过于直觉的规则将会增大后续调整的难度,所以制定分派规则时要作权衡。
\subsection{基本规则}
\begin{asparaenum}
\item EDD(最早交货期)
\suspend{asparaenum}

EDD规则从工期角度出发,将作业按照交货时刻的先后进行排序,并按这个顺序进行生产处理。对于并行机环境,一旦某处理单元空闲,就可以即刻安排队列中的首个作业任务。这个规则适用于安排目标和工期相关的调度任务。
\resume{asparaenum}
\item WSPT(加权最短处理时间)
\suspend{asparaenum}

WSPT规则是SPT(最短处理时间)规则的一般化,从作业的处理时间出发,对以完工时刻为目标的调度任务较为合适。这个规则将处理作业按照$w_j/p_j$值非增的顺序排列,处理时间较长的作业被安排在较后的位置,从一定程度上减少了排队等待时间,并且可以证明WSPT规则对$1\mid \mid \sum w_jC_j$的调度是最优的\cite{pinedo},然而在$Pm \mid \mid \sum w_jC_j$ 环境下并不一定是最优。

\resume{asparaenum}
\item MS(最小松弛)
\end{asparaenum}

MS规则通过描述作业的紧迫程度来进行作业调度,与前两者最大的区别在于,这个规则是动态的,即和系统时间$t$相关。作业根据$\max (d_j - p_j - t , 0)$的值非减的顺序排列,显然较为紧迫的作业会被安排在前面,并且不同的系统时间会影响排列的顺序,呈现动态的调度。该规则适用于安排目标和工期相关的调度任务。
\subsection{复合规则}
复合分派规则是综合了许多基本规则的一个表达式,各基本规则都有其各自的比例参数,用来确定这个规则对符合规则影响程度的比例,没有固定的形式,接下来举一例:ATC(明显滞后成本)规则。

ATC规则综合了WSPT规则和MS规则,每当有空闲处理单元时,所有待调度作业按\eqref{equ:orderindexexample}计算其排序指数,选出具有最大指数值的作业进行处理。
\begin{equation}
I_j(t) = \frac{w_j}{p_j}\exp\left(-\frac{\max(d_j - p_j - t, 0)}{K\bar p}\right) \label{equ:orderindexexample}
\end{equation}
式中:

\begin{tabular}{ll}
$K$ & 规则比例参数\\
$\bar p$ &剩余作业平均处理时间
\end{tabular}

可以看出当$K \to \infty$时,$I_j(t) \to w_j/p_j$,此时ATC规则便转化为WSPT规则。当$K \to 0$时,若作业$j$将产生延期,即$\max(d_j - p_j -t , 0 ) = 0$,那么ATC规则也转化为WSPT规则,若作业不会产生延期,由于$d_j - p_j - t$的影响超过$w_j/p_j$,规则ATC将转化为MS规则。ATC规则可以较容易地应用到$Pm\mid\mid \sum w_jT_j$问题,其关键技术在于比例参数$K$的选取。
\section{基本模型}
基本模型由主要目标函数\eqref{equ:objmain}、次要目标函数\eqref{equ:objsecond}和约束条件\eqref{equ:basicst1} -- (\ref{equ:basicst7})组成,其
\subsection{构造型算法}

\subsection{改进型算法}
\section{插单模型}
\subsection{构造型算法}

\subsection{改进型算法}
\section{分单模型}
\subsection{构造型算法}

\subsection{改进型算法}
\section{综合模型}
\subsection{构造型算法}

\subsection{改进型算法}

\section{小结}
